% Messwerte: Alle gemessenen Größen tabellarisch darstellen
% Auswertung: Berechnung geforderter Ergebnisse mit Schritten/Fehlerformeln/Erläuterung/Grafik (Programme)
\section{Auswertung}
\label{sec:auswertung}

\subsection{Gerätekonstanten}

\subsubsection{Winkelrichtgröße}

Nach \eqref{eqn:richtgröße} folgt $D = \input{build/D.tex}$ als Ergebnis aus den Daten in Tabelle \ref{tab:mess_statisch}.

\begin{table}
	\centering
	\captionsetup{width=0.8\linewidth}
	\caption{Ergebnisse der statischen Messmethode mit einem Achsenabstand von
			 $r = \protect\input{build/a_statisch.tex}$ zur Bestimmung der Winkelrichtgröße.}
	\input{build/mess_tab_statisch.tex}
	\label{tab:mess_statisch}
\end{table}

Dieser Wert setzt sich sich aus dem Mittelwert $\overline{x}$ mit der Rechenvorschrift
\begin{equation*}
	\overline{x} = \pfrac{1}{N \,} \sum_{n=1}^N x_n
	\label{eqn:mittel}
\end{equation*}
und der Standardabweichung $\symup{\Delta}\overline{x}$ zusammen, welche sich über den Ausdruck
\begin{equation*}
	(\symup{\Delta}\overline{x})^2 = \pfrac{1}{N(N-1)} \sum_{n=1}^N (x_n \! - \overline{x})^2
	\label{eqn:std}
\end{equation*}
ermitteln lässt. Unter Python \cite{python} kann dies mithilfe von NumPy \cite{numpy} durch die
Funktionen \verb+numpy.mean+ und \verb+numpy.std+ automatisiert werden.

\subsubsection{Eigenträgheitsmoment}

\begin{table}[H]
	\centering
	\caption{Maße der verwendeten Testmassen mit Messgeräteabweichung.}
	\input{build/dim_tab_test.tex}
	\label{tab:dim_test}
\end{table}

Aufgrund der aus Tabelle \ref{tab:dim_test} entnommenen Messergebnisse wird von identischen Testmassen
mit $m = \input{build/m.tex}$ ausgegangen. Es handelt sich also um Zylinder, welche durch die Längen
$R = 17\hspace{-0.15ex},\hspace{-0.3ex}4 \hspace{0.375ex}\unit{\milli\meter}$ und $h = \input{build/h.tex}$
vollständig beschrieben sind. Im Aufbau~stehen deren Rotationsachsen senkrecht zur Symmetrie, über
Formel~\eqref{eqn:zylinder_senk} lässt sich demnach $I_0 = \input{build/I_0.tex}$ bestimmen. Hier wird die
Abweichung $\symup{\Delta}f$ unter Anwendung der Gaußschen Fehlerfortpflanzung
\begin{equation*}
	(\symup{\Delta}f)^2 = \sum_{n=1}^N
	\left( \! \pfrac{\partial^{\!} f}{\partial x_{\raisebox{0.2ex}{$\scriptstyle{n}$}}} \!
	\right)^{\!\!\! 2} \!\! (\symup{\Delta}x_{\raisebox{0.2ex}{$\scriptstyle{n}$}})^2
	\label{eqn:gauss}
\end{equation*}
bestimmt. Für die nachfolgenden Abschnitte lässt sich dieser Schritt durch Einsatz von uncertainties~\cite{uncertainties}
berücksichtigen. Bei Verschiebung entlang der eingespannten Stange ermittelt sich das Trägheitsmoment
$I_{\hspace{-0.3ex}Z}$ eines einzelnen Zylinders durch \eqref{eqn:steiner} nach Steiner. Das Gesamtträgheitsmoment $I$
des Messaufbaus entspricht nun der Summe des kostanten Eigenträgheitsmoments $I_{\hspace{-0.3ex}D}$ mit denen der
beiden Zylindermassen. Quadrieren von und anschließendes Einsetzen in Gleichung~\eqref{eqn:period} liefert die Beziehung
\begin{equation*}
	T^{\hspace{0.15ex}2} = 4\pi^2 \pfrac{I}{\hspace{-0.15ex}D\hspace{0.15ex}} =
	4\pi^2 \pfrac{1}{\hspace{-0.15ex}D\hspace{0.15ex}}
	\left( 2I_{\hspace{-0.3ex}Z} + I_{\hspace{-0.3ex}D} \right) =
	8\pi^2 \pfrac{1}{\hspace{-0.15ex}D\hspace{0.15ex}} \, ma^2 \! +
	4\pi^2 \pfrac{1}{\hspace{-0.15ex}D\hspace{0.15ex}}
	\left( 2I_0 + I_{\hspace{-0.3ex}D} \right)
\end{equation*}
als Ansatz. Dieser stellt so einen linearen Zusammenhang zwischen Periodendauer $T^{\hspace{0.15ex}2}$ und
zugehörigem Achsenabstand $a^2$ her. Tabelle \ref{tab:mess_dynamisch} enthält die entsprechenden Messungen.

\vfill

\begin{table}
	\centering
	\captionsetup{width=0.775\linewidth}
	\caption{Ergebnisse der dynamischen Messmethode mit einer Auslenkung von
			 $\varphi = \protect\input{build/phi_dynamisch.tex}$ zur linearen Regressionsrechnung.}
	\input{build/mess_tab_dynamisch.tex}
	\label{tab:mess_dynamisch}
\end{table}

\vfill\newpage

Entlang der aufgenommenen Werte wird eine Modellfunktion der Form
\begin{equation*}
	T^{\hspace{0.15ex}2} \hspace{-0.15ex} = p\hspace{0.15ex}a^2 \hspace{-0.15ex} + q
\end{equation*}
zur linearen Regression angewendet. Diese beschreibt eine Gerade $y = px + q$ mit
\begin{align*}
	p = \pfrac{\overline{x\hspace{0.15ex}y\hspace{0.15ex}} - \overline{x} \: \overline{y\hspace{0.15ex}}}
	{\overline{x^2} - \overline{x}^{\hspace{0.15ex}2}} && q = \overline{y\hspace{0.15ex}} - p \overline{x}
	\label{eqn:regression}
\end{align*}
als zugehörige Parameter. Durch Koeffizientenvergleich kann über
\begin{equation*}
	q = 4\pi^2 \pfrac{1}{\hspace{-0.15ex}D\hspace{0.15ex}} \left( 2I_0 + I_{\hspace{-0.3ex}D} \right)
\end{equation*}
das Eigenträgheitsmoment der Drillachse mit der bekannten Winkelrichtgröße zu
\begin{equation*}
	I_{\hspace{-0.3ex}D} = \pfrac{1}{\raisebox{-0.3ex}{$4\pi^2\hspace{-0.15ex}$}} \hspace{0.15ex} D\hspace{0.05ex}q - 2I_0
\end{equation*}
bestimmt werden. Abbildung \ref{fig:plot} reproduziert die resultierende Ausgleichsgerade.

\begin{figure}[H]
	\includegraphics{build/plot.pdf}
	\caption{Quadrierte Messwerte mit passender Gerade aus der linearen Regression.
			 Die Grafik wird mittels Matplotlib \cite{matplotlib} erstellt.}
	\label{fig:plot}
\end{figure}

Die Funktion \verb+numpy.polyfit+ aus der Bibliothek NumPy \cite{numpy} dient zur Berechnung der gesuchten
Regressionsparameter, deren Abweichung aus den jeweiligen Diagonalelementen der Kovarianzmatrix ermittelt wird.
Mit den zuvor definierten Faktoren
\begin{align*}
	p = \input{build/p_dynamisch} && q = \input{build/q_dynamisch}
\end{align*}
ist eine optimale Näherung an die Messdaten gegeben. Daraus ergibt sich schließlich der Wert
$I_{\hspace{-0.3ex}D} = \input{build/I_D.tex}$ als weitere Apparatekonstante.

\subsection{Einfache Körper}

Zur Untersuchung der in Tabelle \ref{tab:dim_körper} charakterisierten Körper fällt die Drehachse mit deren
Symmetrieachsen zusammen. Indem $I_{\hspace{-0.3ex}D}$ von \eqref{eqn:träge_periode} subtrahiert wird, lässt
sich dann aus der Schwingungsdauer experimentell das Trägheitsmoment bestimmen.

\begin{table}
	\centering
	\caption{Maße der verwendeten Körper mit Messgeräteabweichung.}
	\input{build/dim_tab_körper.tex}
	\label{tab:dim_körper}
\end{table}

Um einen Vergleich zu ermöglichen werden zunächst die theoretischen Werte ermittelt. Nach Formel~\eqref{eqn:kugel}
beträgt $I'_{\hspace{-0.3ex}H\hspace{-0.15ex}K} = \input{build/I_kgl_theo.tex}$ für die Holzkugel. Analog dazu folgt
$I'_{\hspace{-0.3ex}H\hspace{-0.15ex}Z} = \input{build/I_zln_theo.tex}$ aus \eqref{eqn:zylinder_symm} im Falle
des Holzzylinders.

\begin{table}
	\centering
	\captionsetup{width=0.65\linewidth}
	\caption{Messwerte der Schwingungsdauer für die Holzkörper bei
			 $\varphi = \protect\input{build/phi_dynamisch.tex}$ als Auslenkung.}
	\input{build/period_tab_körper.tex}
	\label{tab:period_körper}
\end{table}

Aus den gemessenen Schwingungszeiten in Tabelle \ref{tab:period_körper} ergeben sich
$T_{\hspace{-0.15ex}H\hspace{-0.15ex}K} = \input{build/T_kgl.tex}$ und
$T_{\hspace{-0.15ex}H\hspace{-0.15ex}Z} = \input{build/T_zln.tex}$ als gemittelte Ergebnisse. Unter Berücksichtigung
des zuvor bestimmten Eigenträgheitsmoments liefern diese Werte
$I_{\hspace{-0.3ex}H\hspace{-0.15ex}K} = \input{build/I_kgl_exp.tex}$ und
$I_{\hspace{-0.3ex}H\hspace{-0.15ex}Z} = \input{build/I_zln_exp.tex}$ als Resultate. Die Ergebnisse
müssen allerdings verworfen werden, da physikalisch keine negativen Trägheitsmomente auftreten können. Auf die
Gültigkeit dieser Messungen wird genauer in der \hyperref[sec:diskussion]{Diskussion} eingegangen.

\newpage

\subsection{Modellfigur}

Wie in Tabelle \ref{tab:mess_puppe} aufgeführt, werden die Durchmesser $d$ der Körperteile je mehrfach~entlang deren gesamter
Länge gemessen. Anhand dieser leiten sich anschließend die gemittelten Werte zur zylindrischen Approximation ab.

\begin{table}[H]
	\centering
	\caption{Messwerte der zylindrisch genäherten Körperteile der Puppe.}
	\input{build/mess_tab_puppe.tex}
	\label{tab:mess_puppe}
\end{table}

Mit den so beschriebenen Zylindern wird für Tabelle \ref{tab:dim_puppe} über den Volumenanteil auch die entsprechende Masse
ermittelt.

\begin{table}
	\centering
	\captionsetup{width=0.875\linewidth}
	\caption{Gemittelte Maße der Körperteile mit $m = \protect\input{build/m_puppe.tex}$ als Gesamtmasse und
			 $V = \protect\input{build/V_puppe.tex}$ als Gesamtvolumen. Daraus lässt sich die Dichte der Puppe
			 zu $\rho = \protect\input{build/rho_puppe.tex}$ bestimmen.}
	\input{build/dim_tab_puppe.tex}
	\label{tab:dim_puppe}
\end{table}

In der theoretischen Näherung liegt die Symmetrieachse von Kopf und Torso für beide Haltungen der Puppe auf der Drehachse,
es kann also einfach \eqref{eqn:zylinder_symm} verwendet werden. Bei der ersten Pose stehen die Beine ebenfalls parallel zur
Rotation mit einer seitlichen Verschiebung von $R_\textsc{Beine}$ zur Achse. Die Arme sind dagegen orthogonal zur Drehung
vom Körper gestreckt, über Ausdruck \eqref{eqn:zylinder_senk} berechnet sich bei einem seitlichen Achsenabstand entsprechend
der Summe von $R_\textsc{Torso}$ und der halben Höhe $h_\textsc{\hspace{-0.15ex}Arme}$ ihr Trägheitsmoment.~Es ergibt sich
daraus $I'_1 = \input{build/I_1_theo.tex}$ für die gesamte Figur.

Wird nun die zweite Haltung angenommen, bleiben alle bestehenden Distanzen erhalten. Durch die veränderte Orientierung
der Beine kommt allerdings noch eine Verschiebung der Achse um deren halbe Höhe $h_\textsc{Beine}$ nach vorn hinzu.
Über den Hypotenusensatz nach Pythagoras ergibt sich der angepasste Gesamtabstand, welcher in den Steinerschen
Satz~\eqref{eqn:steiner} eingesetzt werden muss. Das Trägheitsmoment lässt sich damit schließlich zu
$I'_2 = \input{build/I_2_theo.tex}$ bestimmen.

\begin{table}
	\centering
	\captionsetup{width=0.725\linewidth}
	\caption{Messwerte der Schwingungsdauer für zwei Körperhaltungen unter je zwei Auslenkungen.}
	\input{build/period_tab_puppe.tex}
	\label{tab:period_puppe}
\end{table}

Aus den in Tabelle \ref{tab:period_puppe} nachgehaltenen Periodendauern ergeben sich für beide Posen die
jeweiligen Zeiten $T_1 = \input{build/T_1.tex}$ und $T_2 = \input{build/T_2.tex}$ als mittlere Messungen, mit deren Hilfe
die experimentelle Bestimmung der Trägheitsmomente unter Verwendung von Ausdruck \eqref{eqn:träge_periode} erfolgt. Dieses
Vorgehen liefert so $I_1 = \input{build/I_1_exp.tex}$ und $I_2 = \input{build/I_2_exp.tex}$ als resultierende Werte.
Wie zuvor erwähnt dürfen keine negativen Ergebnisse auftreten. Ein sinnvoller Korrekturansatz folgt.
\enlargethispage{\baselineskip}\newpage

