% Diskussion: Resultate mit Fehler/Genauigkeit zusammenstellen, Literaturwerte/Messmethoden/Ursachen vergleichen
% Literatur: Verwendete Literatur/Grafiken/Werte/Programme
% Anhang: Kopie der analog eingetragenen Messdaten
\section{Diskussion}
\label{sec:diskussion}

Negative Trägheitsmomente sind physikalisch nicht möglich. Um diesen somit signifikanten Fehler zu behandeln,
welcher hier für jeden untersuchten Körper auftritt, wird nach einer plausiblen Fehlerquelle gesucht. Als wahrscheinlichster
Kandidat fällt der Blick dabei auf die bei der Bestimmung der Gerätekonstanten verwendete Stange. Diese kann durch einen
Zylinder vernachlässigbarer Dicke mit einer Höhe $h = \input{build/h_stab.tex}$ genähert werden. Für eine Masse
$m = \input{build/m_stab.tex}$ ergibt sich nach \eqref{eqn:zylinder_senk} der Wert \mbox{$I_{\hspace{-0.15ex}S} =
(287\hspace{-0.15ex},\hspace{-0.3ex}4 \pm 0,\hspace{-0.3ex}3) \cdot\qty{e-5}{\kilo\gram\meter\squared}$}
als ihr Beitrag, welcher innerhalb der gleichen Größenordnung wie das zuvor ermittelte Eigenträgheitsmoment
$I_{\hspace{-0.3ex}D} = \input{build/I_D.tex}$ liegt. Dieser darf daher bei dessen Berechnung nicht ausgelassen werden.
Unter Beachtung dieser Erkenntnis lässt sich die korrigierte Konstante von gerade einmal
$I_{\hspace{-0.3ex}D} = \input{build/I_D_korr.tex}$ aufstellen. Wird diese für eine wiederholte Berechnung
der Trägheitsmomente herangezogen, ergeben sich die Werte
$I_{\hspace{-0.3ex}H\hspace{-0.15ex}K} = \input{build/I_kgl_exp_korr.tex}$ und
$I_{\hspace{-0.3ex}H\hspace{-0.15ex}Z} = \input{build/I_zln_exp_korr.tex}$ für die einfachen Holzkörper.
Der Vergleich mit den zuvor gewonnenen theoretischen Vorhersagen
$I'_{\hspace{-0.3ex}H\hspace{-0.15ex}K} = \input{build/I_kgl_theo.tex}$ und
$I'_{\hspace{-0.3ex}H\hspace{-0.15ex}Z} = \input{build/I_zln_theo.tex}$ gibt entsprechende relative Übereinstimmungen
von $\input{build/rel_kgl.tex}$ und $\input{build/rel_zln.tex}$ zwischen Experiment und Modell an.
Verbleibende Abweichungen erklären sich durch weitere Einflüsse. So werden etwa die Massen der Halterungen
vernachlässigt. Da sich diese hauptsächlich auf der Rotationsachse befinden, sollte der resultierende Fehler jedoch minimal
ausfallen. Generelle Ungenauigkeiten beim Messverfahren sind ebenfalls nicht ausgeschlossen. Bei der Rückstellkraftmessung
zur Bestimmung der Winkelrichtgröße muss beispielsweise genau senkrecht angesetzt werden, ansonsten kann es zu verfälschten
Ergebnissen kommen.


Die verschiedenen Haltungen der Modellfigur belaufen sich mithilfe dieser Korrektur auf
$I_1 = \input{build/I_1_exp_korr.tex}$ sowie $I_2 = \input{build/I_2_exp_korr.tex}$ als experimentelle Resultate. Dies
entspricht etwa $\input{build/rel_1.tex}$ von $I'_1 = \input{build/I_1_theo.tex}$ sowie $\input{build/rel_2.tex}$ von
$I'_2 = \input{build/I_2_theo.tex}$ als Theoriewerte. In der zweiten Pose decken sich mittlere Vorhersage und Ergebnis
demnach zu einem sehr hohen Grad. Dagegen liegt das erwartete theoretische Trägheitsmoment für die erste Haltung eine
volle Größenordnung über dem resultierenden Mittelwert. Erklären lässt sich dieser Umstand mit den breiten Fehlerintervallen
der Messungen. Diese wiederum begründen sich vor allem darin, dass die Puppe in beiden Posen Trägheitsmomente aufweist, die
sich im Bereich des Eigenbeitrags $I_{\hspace{-0.3ex}D} = \input{build/I_D_korr.tex}$ der Apparatur befinden. Deren
Bestimmung wird dadurch deutlich störanfälliger. Abgesehen von den vorherigen Fehlerquellen tragen hier auch noch die
ungenaue Positionierung der Körperteile sowie die vereinfachte Geometrie zur realen Abweichung bezüglich der
Theorievorhersage bei.

Zuletzt folgt aus der Modellierung der Figur in der ersten Haltung ein Trägheitsmoment entsprechend
$\input{build/rel_theo.tex}$ der zweiten Pose. Diese Vorhersage ist erwartet, da die vom Körper weggestreckten Beine im
zweiten Fall für eine Massenverteilung in weiterer Entfernung zur Drehachse sorgen. Im Experiment beträgt das
Trägheitsmoment in der ersten nur $\input{build/rel_exp.tex}$ dessen in der zweiten Haltung. Werden die großen
Fehlerintervalle beachtet, dürfen theoretische und reale Änderung an dieser Stelle aber trotzdem als vergleichbar behandelt
werden. Sie sind daher wenigstens mit dem Satz von Steiner vereinbar.
\enlargethispage{\baselineskip}\newpage

