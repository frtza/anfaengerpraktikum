% Aufgabe: Messaufgaben auflisten
% Vorbereitung: Vorbereitungsaufgaben bearbeiten
% Versuchsaufbau: Verwendete Apparatur, Beschreibung Funktionsweise/Nutzen mit Skizze/Foto
\newpage

\section{Durchführung}
\label{sec:durchführung}

Für die Untersuchung verschiedener Trägheitsmomente wird eine Drillachse verwendet. Diese ist zweifach drehbar gelagert
und über eine Spiralfeder fest mit einem Rahmen verbunden. An ihrem oberen Ende lassen sich die zu betrachtenden Körper
fixieren. Der Messapparat ist durch zwei Gerätekonstanten charakterisiert, welche es zu bestimmen gilt. Zunächst lässt
sich die Winkelrichtgröße $D$ anhand der \textit{statischen Messmethode}~ermitteln. Dazu wird eine Stange
vernachlässigbarer Masse waagerecht eingespannt. Anschließend wird an diese ein Schraubfederkraftmesser
in konstantem Abstand senkrecht angesetzt und die Rückstellkraft für verschiedene Auslenkwinkel aufgenommen.
Weiter lässt sich das Eigenträgheitsmoment $I_{\hspace{-0.3ex}D}$ der Drillachse über die \textit{dynamische Messmethode}
feststellen, indem nun zwei zylindrische Massen symmetrisch an den beiden Enden der Stange~befestigt werden. Unter Variation
des Abstands wird die Apparatur in Schwingung versetzt und die Periodendauer notiert. Messen der fünffachen Zeit und
anschließendes Mitteln sorgen dabei für eine Reduktion fehlerhafter Einflüsse.

Sind die Gerätekonstanten bekannt, wird mit den eigentlichen Messungen fortgefahren. Mit einer Kugel und einem Zylinder
werden dabei die Trägheitsmomente einfacher Körper betrachtet. Nach initialer Auslenkung lassen sich diese erneut über die
Schwingungsperiode ermitteln, deren Bestimmung wieder nach der vorherigen Beschreibung erfolgt. Analoges Vorgehen
liefert den Wert einer geometrisch komplexeren Modellpuppe. Diese wird in zwei Körperhaltungen bemessen. In der ersten
Haltung befinden sich die Arme seitlich vom Körper gestreckt, die Beine stehen senkrecht unter dem Torso. Die zweite Haltung
behält die Position der Arme bei, die Beine zeigen nun parallel nach vorn. Zur theoretischen Näherung werden Kopf, Arme,
Torso und Beine der Figur als Zylinder genähert.

\newpage
