% Titel: Ziel des Versuchs
% Theorie: Physikalische Grundlagen von Versuch/Messverfahren, Gleichungen ohne Herleitung knapp erklären
\section{Zielsetzung}

Im nachfolgenden Versuch sollen die Trägheitsmomente von verschiedenen Körpern unter Verwendung einer Drillachse
bestimmt werden. Durch Vergleich mit den vorhergesagten Werten wird außerdem der Steinersche Satz verifiziert.
Alle theoretischen Grundlagen sind der Versuchsanleitung \cite{träge} entnommen.

\section{Theorie}
\label{sec:theorie}

% \subsection{Rotationsbewegungen}

Allgemein lässt sich die Dynamik eines rotierenden Systems vollständig durch das wirkende Drehmoment $\symbf M$,
den Trägheitstensor $\symbf I$ und die Winkelbeschleunigung $\symbf{\ddot{\varphi}}$ \mbox{beschreiben. Um} das
Trägheitsmoment einer punktförmigen Masse $m$ im Abstand $r$ zur Drehachse für den skalaren Fall zu ermitteln, wird
der Ausdruck
\begin{equation}
	I = mr^2
	\label{eqn:punkt}
\end{equation}
herangezogen. Rotiert nun ein ausgedehnter starrer Körper mit Volumen $V$ und konstanter Dichte $\rho$ um eine
feste Achse gilt, dass sich alle infinitesimalen Massen $\symup dm$ mit identischer Winkelgeschwindigkeit $\dot{\varphi}$
bewegen. Aus \eqref{eqn:punkt} kann damit der Term
\begin{equation}
	I = \!\!\int\! r^2 \, \symup dm = \!\!\int\! r^2 \rho \, \symup dV
	\label{eqn:träge}
\end{equation}
abgeleitet werden. Eine weitere vereinfachende Beziehung liefert der Steinersche Satz. Dieser besagt, dass sich das
Trägheitsmoment eines Körpers der Masse $m$ bezüglich einer Achse mit Abstand $a$ zur parallelen Schwerpunktsachse mit
Trägheitsmoment $I_0$ über
\begin{equation}
	I = I_0 + ma^2
	\label{eqn:steiner}
\end{equation}
berechnet. Unter Verwendung von Formel~\eqref{eqn:träge} können gerade für einfache Geometrien gut die speziellen
Trägheitsmomente hinsichtlich solcher Drehachsen bestimmt werden, die durch deren jeweiligen Schwerpunkt verlaufen.
Die Vorschrift
\begin{equation}
	I_0 = \pfrac{2}{\raisebox{0.25ex}{$5$}} m R^2
	\stepcounter{equation}\tag{4a}
	\label{eqn:kugel}
\end{equation}
gibt dieses für eine Vollkugel mit Radius $R$ an. Für einen Vollzylinder bemisst
\begin{equation}
	I_0 = \pfrac{1}{\raisebox{0.25ex}{$2$}} m R^2
	\tag{4b}
	\label{eqn:zylinder_symm}
\end{equation}
dessen Trägheitsmoment bei Rotation um die Symmetrieachse. Mit
\begin{equation}
	I_0 = m \left( \pfrac{1}{\raisebox{0.25ex}{$4$}} R^2 + \pfrac{1}{\raisebox{0.25ex}{$12$}} h^2 \right)
	\stepcounter{equation}\tag{4c}
	\label{eqn:zylinder_senk}
\end{equation}
ist jenes für denselben Zylinder gegeben, wenn die Drehachse senkrecht zur Mantelfläche durch den Schwerpunkt verläuft.
Dazu wird nun auch die Höhe $h$ benötigt.
\enlargethispage{\baselineskip}\newpage

Beim Ansetzen einer Kraft $\symbf F$ im Abstand $\symbf r$ zur Achse wird das Drehmoment über den Zusammenhang
$\symbf M = \symbf F \times \symbf r$ definiert. Durch die bekannte Proportionalität $M \propto \sin\theta$ erreicht
der Betrag des Vektorprodukts unter orthogonalem Einwirken bei $\theta = \qty{90}{\degree}$
\begin{equation}
	M = F \hspace{0.3ex} r
	\label{eqn:dreh_kraft}
\end{equation}
als Maximum. Analog zur Translation mit $F=m\hspace{0.15ex}\ddot{r}$ kann dieses als $M = I \,\ddot{\varphi}$ formuliert
werden. Handelt es sich um ein schwingungsfähiges System, wirkt der Drehung außerdem ein in erster Näherung zum Auslenkwinkel
$\varphi$ lineares Rückstellmoment
\begin{equation}
	M = -D \,\varphi
	\label{eqn:dreh_winkel}
\end{equation}
mit $D$ als Winkelrichtgröße entgegen. Gleichsetzen der Beträge von \eqref{eqn:dreh_kraft}
und \eqref{eqn:dreh_winkel} ergibt
\begin{equation}
	D = \pfrac{F\hspace{0.3ex}r}{\raisebox{0.6ex}{$\varphi$}}
	\label{eqn:richtgröße}
\end{equation}
als Beziehung, mit deren Hilfe der benannte Faktor bestimmt werden kann. Aufstellen des Terms
$I \,\ddot{\varphi} = -D \,\varphi$ liefert dann genau die charakteristische Bewegungsgleichung des harmonischen
Oszillators, dessen Schwingungsdauer
\begin{equation}
	T = \pfrac{1}{\raisebox{0.6ex}{$\nu\hspace{0.375ex}$}} =
	\pfrac{2\pi}{\raisebox{0.6ex}{$\hspace{-0.15ex}\omega$}} =
	2\pi \sqrt{\hspace{-0.15ex}\pfrac{I}{\hspace{-0.15ex}D\hspace{0.15ex}}\hspace{0.15ex}}
	\label{eqn:period}
\end{equation}
sich über Frequenz $\nu$ und Kreisfrequenz $\omega$ bestimmt. Bei Drehschwingungen gilt diese zwar nur in
Kleinwinkelnäherung, bietet sich durch Umstellen nach
\begin{equation}
	I = D \,\pfrac{T^{\hspace{0.15ex}2}}{\raisebox{-0.3ex}{$4\pi^2$}}
	\label{eqn:träge_periode}
\end{equation}
aber trotzdem zur Bestimmung des Trägheitsmomentes an.

