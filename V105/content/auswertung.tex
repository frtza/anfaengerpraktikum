% Messwerte: Alle gemessenen Größen tabellarisch darstellen
% Auswertung: Berechnung geforderter Ergebnisse mit Schritten/Fehlerformeln/Erläuterung/Grafik (Programme)
\section{Auswertung}
\label{sec:auswertung}

Im folgenden Abschnitt wird die Fehlerrechnung mithilfe der Bibliothek uncertainties~\cite{uncertainties} unter Python
\cite{python} automatisiert. Grafische Darstellungen lassen sich mittels Matplotlib~\cite{matplotlib} erstellen, während
NumPy~\cite{numpy} mittels der Funktion \verb+numpy.polyfit+ die lineare Regression durchführt. Abweichungen werden
dabei aus der Kovarianzmatrix entnommen.

\subsection{Apparatur}

\begin{figure}[H]
	\vspace{-1em}
	\includegraphics{build/plot_feld.pdf}
	\vspace{-4ex}
	\captionsetup{width=0.86\linewidth}
	\caption{Magnetfeldstärke und ortsbedingte Änderungsrate mit $N = \num{195}$, \newline
			 $d = \qty{0.138}{\meter}$, $R = \qty{0.109}{\meter}$ und $I = \qty{1}{\ampere}$ als Parameter der Apparatur.}
	\label{fig:feld}
\end{figure}

Anhand von Abbildung \ref{fig:feld} lässt sich graphisch verifizieren, dass die Feldstärke \eqref{eqn:feld_helmholtz} zwischen
den beiden Spulen mit $N = \num{195}$, $d = \qty{0.138}{\meter}$ und $R = \qty{0.109}{\meter}$ weitestgehend konstant bleibt.
Auch der Feldgradient entlang der Symmetrieachse, welcher sich nach \eqref{eqn:grad_spule} berechnet, ist in diesem
Bereich vernachlässigbar. Es wird daher die Annahme getroffen, dass sich das Magnetfeld, dessen Wirkung hier weiterführend
ausgenutzt werden soll, homogen verhält und über Term~\eqref{eqn:feld_helmholtz} bei $x = 0$ bestimmen lässt. Mit
$I = \qty{1}{\ampere}$ folgt dann $B = 1\hspace{-0.1ex},\hspace{-0.3ex}356 \cdot\hspace{-0.1ex} \qty{e-3}{\tesla}$ als
Faktor, der anschließend zur Ermittlung der magnetischen Flussdichte bei entsprechenden Vielfachen dieser Stromstärke dient.

Für die dynamischen Messmethoden wird außerdem das Trägheitsmoment des Testkörpers benötigt. Dieser wird als
Vollkugel homogener Dichte approximiert, womit sich
\begin{equation*}
	J = \pfrac{2}{\raisebox{0.15ex}{$5$}} M R^2
\end{equation*}
ergibt. Mit Radius $R = \qty{2.5}{\centi\meter}$ und Masse $M = \qty{150}{\gram}$ folgt aus diesem Ausdruck der Wert
$J = 3,\hspace{-0.6ex}75 \cdot\hspace{-0.1ex} \qty{e-5}{\kilo\gram\meter\squared}$ als Trägheitsmoment.

\subsection{Gravitation}

Die Messungen aus Tabelle \ref{tab:grav} werden in Abbildung \ref{fig:grav} gegeneinander aufgetragen. Ebenfalls
dargestellt ist die entsprechende lineare Regressionsfunktion.

\begin{table}[H]
	\centering
	\caption{Messdaten zur statischen Methode.}
	\input{build/tab_grav.tex}
	\label{tab:grav}
\end{table}
\begin{figure}[H]
	\vspace{-1em}
	\includegraphics{build/plot_grav.pdf}
	\vspace{-4ex}
	\caption{Datenpunkte mit linearer Näherungskurve.}
	\label{fig:grav}
\end{figure}

Um eine möglichst gute Übereinstimmung mit den aufgenommenen Daten zu erzielen, werden beide Parameter der Gerade
$y = ax + b$ genutzt. Die konstante Verschiebung fällt zwar bei der weiteren Verarbeitung weg, dient hier aber
als zusätzlicher Freiheitsgrad.

Mit den Diagonalelementen der Kovarianzmatrix liefert dieser Regressionsansatz
\begin{align*}
	a = \input{build/a_grav.tex} && b = \input{build/b_grav.tex}
\end{align*}
als fehlerbehaftete Koeffizienten. Vergleich mit \eqref{eqn:dreh_gleichgewicht_skalar} lässt den Zusammenhang
\begin{equation*}
	a = \pfrac{\mu}{\raisebox{0.6ex}{$mg$}}
\end{equation*}
erkennen. Daraus folgt mit der punktförmig genäherten Testmasse von $m = \qty{1.4}{\gram}$ der~Wert
$\mu = \input{build/mu_grav}$ für das magnetische Moment des kugelförmigen Dipols.

\subsection{Schwingung}

Analoges Vorgehen produziert die in Abbildung \ref{fig:schw} einsehbare graphische Darstellung. Es werden
nun allerdings bereits modifizierte Messdaten aus Tabelle \ref{tab:schw} in Form der quadrierten Periodendauer
gegen den Kehrwert der magnetischen Flussdichte angezeigt.

\begin{figure}[H]
	\includegraphics{build/plot_schw.pdf}
	\vspace{-4ex}
	\caption{Datenpunkte mit linearer Näherungskurve.}
	\label{fig:schw}
\end{figure}

Durch die auf diese Weise erhaltenen Parameter der linearen Regressionsrechnung
\begin{align*}
	a = \input{build/a_schw.tex} && b = \input{build/b_schw.tex}
\end{align*}
werden die modellspezifischen Fehlerquadrate minimal.

\begin{table}[H]
	\centering
	\caption{Messdaten zur dynamischen Oszillatormethode.}
	\input{build/tab_schw.tex}
	\label{tab:schw}
\end{table}

Aus Gleichung~\eqref{eqn:schwingung} lässt sich über die Schwingunsdauer eine Beziehung der Form
\begin{equation*}
	a = \pfrac{4\pi^2J}{\raisebox{0.6ex}{$\mu$}}
\end{equation*}
herleiten. Mithilfe der in Tabelle \ref{tab:schw} ersichtlichen Messungen wird das magnetische Moment dann
zu $\mu = \input{build/mu_schw.tex}$ bestimmt.

\subsection{Präzession}

Wie zuvor werden die Messwerte in Tabelle \ref{tab:präz} zur Ausgleichsrechnung herangezogen.

\begin{table}[H]
	\centering
	\caption{Messdaten zur dynamischen Kreiselmethode.}
	\input{build/tab_präz.tex}
	\label{tab:präz}
\end{table}

\begin{figure}[H]
	\includegraphics{build/plot_präz.pdf}
	\vspace{-4ex}
	\caption{Gewertete Datenpunkte mit entsprechenden linearen Näherungskurven.}
	\label{fig:präz}
\end{figure}

Hier findet die reziproke Umlaufzeit Verwendung, indem diese gegen die Magnetfeldstärke aufgetragen wird. Für den
Steigungsparameter lässt sich über \eqref{eqn:präzession} der Ausdruck
\begin{equation*}
	a = \pfrac{\mu}{2\pi L_0}
\end{equation*}
formulieren. Vorausgesetzt wird eine Rotationsfrequenz von $\nu = \qty{6}{\hertz}$ mit einem Wert von
$L_0 = \input{build/L_0.tex}$ als Drehimpuls. Die Verifikation mittels der Stroboskopvorrichtung lässt für die
Stromstärken $I = \qty{1.4}{\ampere}$ und $I = \qty{3.4}{\ampere}$ auf eine besonders hohe Genauigkeit schließen.
Abbildung~\ref{fig:präz} hebt diese Daten mit der zugehörigen Schnittgerade durch eine höhere Opazität hervor. Als
Parametrisierung ergeben sich an dieser Stelle die Faktoren
\begin{align*}
	a = \input{build/a_präz_2.tex} && b = \input{build/b_präz_2.tex}
\end{align*}
sowie $\mu = \input{build/mu_präz_2.tex}$ als resultierendes magnetisches Moment der Kugel. Die transparente,
nahezu parallel verlaufende Regressionsgerade berücksichtigt mit den Koeffizienten
\begin{align*}
	a = \input{build/a_präz_1.tex} && b = \input{build/b_präz_1.tex}
\end{align*}
alle aufgezeichneten Messdaten. Mit $\mu = \input{build/mu_präz_1.tex}$ wird das passende Ergebnis bemessen.
Der steilere eingefügte Geradenabschnitt dient zur anschaulichen Bewertung der unterschiedlichen Messansätze
und wird im nächsten Abschnitt behandelt.

