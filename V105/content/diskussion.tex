% Diskussion: Resultate mit Fehler/Genauigkeit zusammenstellen, Literaturwerte/Messmethoden/Ursachen vergleichen
% Literatur: Verwendete Literatur/Grafiken/Werte/Programme
% Anhang: Kopie der analog eingetragenen Messdaten
\section{Diskussion}
\label{sec:diskussion}

Neben den unvermeidbaren Ungenauigkeiten bei der Messwertaufzeichnung lässt sich ein weiterer genereller Fehlereinfluss
durch die trotz verbauter Libelle nicht kompensierte Neigung der Apparatur festhalten. Die resultierenden Abweichungen
dürften dabei jedoch vernachlässigbar sein. Anders verhält es sich mit der Leistung des Luftkissens, welches den Kontakt
zwischen Kugel und Halterung nur in unregelmäßigen Intervallen minimiert. Es kommt zu zeitweise starker Reibung, dies
hat signifikante Verfälschungen der Daten zur Folge. Für die statische Messung lässt sich ein fehlerhaftes Festsetzen
gut von echten Gleichgewichtslagen unterscheiden, weshalb die Ergebnisse hierzu eine vergleichsweise geringe Streuung
aufweisen. Das auf diese Weise ermittelte magnetische Moment lautet $\mu = \input{build/mu_grav.tex}$ und ist mit dem
ungefilterten Wert $\mu = \input{build/mu_präz_1.tex}$ sowie dem gewerteten Resultat $\mu = \input{build/mu_präz_2.tex}$
aus der Präzession vereinbar. Betrachtung der Methode über die Schwingungsdauer zeigt mit $\mu = \input{build/mu_schw}$
eindeutig eine schlechtere Übereinstimmung auf, obwohl ebenfalls ein eher geringer Fehlerbereich vorliegt. Diese
Tatsache kann wahrscheinlich auf erhöhte Reibung durch Fehlfunktion des Luftkissens zurückgeführt werden, deren
Korrektur sich für dynamische Methoden wesentlich problematischer gestaltet. Zur Präzessionsuntersuchung wird ein
baugleicher Apparat genutzt, der recht stabil geringere mechanische Widerstände erzielt. Die größeren
Unsicherheitsintervalle stammen in diesem Fall aus anderer Quelle. Es sei dazu angemerkt, dass idealerweise eine
Rotationsfrequenz zwischen \qty{4}{\hertz} und \qty{6}{\hertz} angestrebt wird, da die zeitlich exponentiell
abfallende Frequenzkurve dazwischen noch ausreichend flach verläuft. Eine gezielt konstante Rotationsrate kann durch
Drehen von Hand allerdings kaum gewährleistet werden. Dies lässt sich an den Umlaufzeiten bei $I = \qty{2.2}{\ampere}$,
$I = \qty{2.4}{\ampere}$ und $I = \qty{2.6}{\ampere}$ in Tabelle \ref{tab:präz} illustrieren, da diese Messwerte
einem gemeinsamen Drehvorgang entspringen. Die in Abbildung \ref{fig:präz} ausschnitthaft dargestellte
Regressionsgerade mit
\begin{align*}
	a = \input{build/a_präz_3.tex} && b = \input{build/b_präz_3.tex}
\end{align*}
passt gut zu den Daten. Besagtes Vorgehen, das Magnetfeld also anzupassen, ohne die Kugel währenddessen zu stoppen,
ist zunächst durchaus brauchbar. Wird daraus für $\qty{6}{\hertz}$ das magnetische Moment bestimmt, ergibt sich mit
$\mu = \input{build/mu_präz_3.tex}$ ein stark ausreißendes Ergebnis. Alternativ lässt sich durch die
Annahme, dass der reale Wert $\mu = \input{build/mu_präz_2.tex}$ entspricht, eine plausible Drehrate
$\nu = \input{build/f_präz_3.tex}$ ermitteln. Diese liegt außerhalb des erlaubten Frequenzbereichs und ist somit
eigentlich ungeeignet. 

Zusammenfassend fällt es schwer, eine der Methoden als allgemein überlegen zu bezeichnen. Unter den gegebenen Bedingungen
erzielt der statische Messansatz unter Einwirken der Gravitation das exakteste Ergebnis. Die beiden dynamischen
Herangehensweisen könnten unter anderen Umständen aber ebenfalls in der Lage sein, Werte äquivalenter oder sogar
höherer Qualität zu liefern.
