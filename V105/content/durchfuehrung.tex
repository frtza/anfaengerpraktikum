% Aufgabe: Messaufgaben auflisten
% Vorbereitung: Vorbereitungsaufgaben bearbeiten
% Versuchsaufbau: Verwendete Apparatur, Beschreibung Funktionsweise/Nutzen mit Skizze/Foto
\section{Durchführung}
\label{sec:durchführung}

Der Versuchsaufbau besteht im Wesentlichen aus einem Helmholtz-Spulenpaar mit vertikal ausgerichteter Symmetrieachse
zur Erzeugung eines externen homogenen Magnetfeldes. Darin ist mittig ein Messingzylinder positioniert. Dieser erzeugt
ein Luftkissen, welches für eine möglichst reibungsfreie Fixierung des zu untersuchenden Körpers sorgt. Dabei handelt
es sich hier um eine Kugel, in der ein zylindrischer Permanentmagnet versenkt ist. Entlang der Richtung des magnetischen
Moments ist auf deren Oberfläche ein kurzer Stiel mit einer Bohrung zum Schwerpunkt des Testobjekts befestigt. Zur
Überprüfung der Rotationsfrequenz befindet sich noch ein Stroboskop oberhalb der Messapparatur. Spulenstrom, Luftkissen
und Stroboskop lassen sich über eine Kontrolleinheit individuell ansteuern. Bei längerer Belastung steigt die Temperatur
und damit der elektrische Widerstand, sodass die maximale Feldstärke nicht mehr erreicht werden kann. Daher sollte
kein Strom fließen, während die Magnetfelder nicht betrieben werden müssen.

Zur statischen Messung wird eine dünne Aluminiumstange, deren Einfluss vernachlässigbar ist, durch die Bohrung am Stiel
in die Kugel eingeführt. Daran befindet sich eine als punktförmig genäherte, verschiebbare Masse. Unter Variation des
Abstandes kann die Stromstärke so eingepegelt werden, dass sich ein Gleichgewicht einstellt. Nachhalten der
entsprechenden Werte erlaubt die anschließende Untersuchung des Dipols.

Darauf folgt die dynamische Methode durch Betrachtung der Periodendauer. Die Kugel wird geringfügig aus der Ruhelage
ausgelenkt, es werden jeweils zehn Schwingungsvorgänge für unterschiedliche Stromstärken aufgenommen. Durch Mitteln
der so erhaltenen Zeiten reduziert dieses Vorgehen mögliche Fehlereinflüsse. Dabei bleibt der Aluminiumstab ohne
die Punktmasse verbunden, um die Oszillation leichter nachvollziehen zu können.

Zuletzt erfolgt die Datenaufnahme über die Präzession. Nun ohne Aluminiumstange, kann durch Einstellen der
geforderten Frequenz am Stroboskop mithilfe einer weißen Kennzeichnung am Stiel die Rotationsrate geprüft werden.
Solange die Markierung im Licht stationär erscheint, stimmen diese überein. Die Rotation der Kugel wird zunächst
stabilisiert, sodass keine Nutationsbewegungen auftreten. Nachdem dies erfolgt ist, kann die Drehachse leicht von
ihrer vertikalen Position ausgelenkt werden. Das Magnetfeld ist bei dieser Methode zunächst abgeschaltet. Wird es
nun aktiviert, lässt sich die erwartete Präzession beobachten. Unter variabler Stromstärke werden die Umlaufzeiten
gemessen.

