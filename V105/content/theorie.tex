% Titel: Ziel des Versuchs
% Theorie: Physikalische Grundlagen von Versuch/Messverfahren, Gleichungen ohne Herleitung knapp erklären
\section{Zielsetzung}

Die nachfolgende Versuchsreihe dient zur Bestimmung des magnetischen Moments eines Permanentmagneten. Es werden
dazu verschiedene Messverfahren angewendet, welche den Testkörper unter Einwirkung eines Drehmoments sowie als
harmonischen Oszillator und präzedierenden Kreisel untersuchen. Die notwendigen theoretischen Grundlagen sind
der Versuchsanleitung \cite{magnetisch} entnommen.

\section{Theorie}
\label{sec:theorie}

\subsection{Homogene Magnetfelder}

Durch einen vom Strom~$I$ durchflossenen Leiter mit dem infinitesimalen Längenelement~$\symup d \symbf s$
wird am Ort~$\symbf r$ nach dem Biot-Savart-Gesetz die magnetische Flussdichte
\begin{equation}
	\symup d \symbf B = \frac{\mu_0}{\raisebox{0.15ex}{$4\pi$}} \hspace{0.15ex} I
	\symup d \symbf s \times \frac{\raisebox{-0.45ex}{$\symbf r$}}{r^3}
	\label{eqn:biot-savart}
\end{equation}
erzeugt. Der Bezeichner $\mu_0 = \input{build/mu_0.tex}$ \cite{scipy} gibt hier die magnetische
Permeabilität des Vakuums an. Auf der Symmetrieachse einer aus $\hspace{-0.1ex}N\hspace{0.1ex}$ kreisförmigen
Leiterschleifen mit Radius~$R$ zusammengesetzten Spule ist damit ein Feld der Form
\begin{equation}
	\symbf B_0(x) = \frac{\mu_0}{\raisebox{0.15ex}{$2$}} \hspace{0.015ex}N\hspace{-0.015ex}I\hspace{0.15ex}
	\frac{\raisebox{-0.45ex}{$R^2$}}{(R^2 + x^2)^{3\hspace{-0.045ex}/2}} \hspace{0.15ex} \hat{\symbf x}
	\label{eqn:feld_spule}
\end{equation}
gegeben. Der Feldgradient entlang dieser Achse entspricht dann
\begin{equation}
	\frac{\symup dB_0}{\symup dx} = -\frac{3\mu_0}{\raisebox{0.15ex}{$2$}} \hspace{0.015ex}N\hspace{-0.015ex}I\hspace{0.15ex}
	\frac{\raisebox{-0.45ex}{$R^2x$}}{(R^2 + x^2)^{5\hspace{-0.12ex}/2}}
	\label{eqn:grad_spule}
\end{equation}
und zeigt die örtliche Änderungsrate der Feldstärke~$B_0$ an. Um ein homogenes Magnetfeld herstellen zu können,
werden zwei gleichsinnig vom Strom~$I$ durchflossene Kreisspulen identischer Windungszahl~$N$ im gegenseitigen
Abstand~$d$ so zueinander positioniert, dass sich ihre Symmetrieachsen überlagern. Die Superposition der einzelnen
Beiträge sorgt für eine näherungsweise konstante Feldstärke um den gemeinsamen Mittelpunkt. Liegt dieser im
Ursprung, lässt sich die skalare Flussdichte \eqref{eqn:feld_spule} durch
\begin{equation}
	B(x) = B_0\!\left(x+\pfrac{d}{\raisebox{0.21ex}{$2$}}\right) + B_0\!\left(x-\pfrac{d}{\raisebox{0.21ex}{$2$}}\right)
	\label{eqn:feld_helmholtz}
\end{equation}
ausdrücken. Eine solche Anordnung heißt Helmholtz-Spule, wenn $d = R$ gilt. In diesem Fall wird das zentrale Feld
maximal homogen und variiert mit $B(x) = B(0) + \mathcal{O}(x^4)$ nur in vierter Ordnung. Für abweichende Distanzen kann
mit konstanter linearer Näherung noch $B(x) = B(0) + \mathcal{O}(x^2)$ geschrieben werden. Befindet sich ein Dipol in einem
homogenen Magnetfeld, erfährt dieser ein Drehmoment der Form
\begin{equation}
	\symbf D = \symbf \mu \times \symbf B
	\label{eqn:dreh_magnetisch}
\end{equation}
und richtet sich so aus, dass magnetisches Moment $\symbf \mu$ und Flussdichte $\symbf B$ \mbox{parallel verlaufen.}
\enlargethispage{\baselineskip}\newpage

\subsection{Makroskopische Dipole}

Ein magnetischer Dipol lässt sich beispielsweise durch einen Permanentmagneten oder~eine stromdurchflossene
Leiterschleife realisieren. Letztere weist ein magnetisches Moment
\begin{equation}
	\symbf \mu = I \symbf A
	\label{eqn:moment_schleife}
\end{equation}
auf, welches senkrecht zu ihrer Querschnittsfläche $A$ steht. Für Permanentmagneten ist die rechnerische Bestimmung
von $\symbf \mu$ problematisch, das magnetische Moment kann jedoch auf verschiedene Weisen experimentell ermittelt
werden.

\subsubsection{Gravitation}

Als statische Methode wird die Gravitationskraft $\symbf F = m \symbf g$ mit der Erdbeschleunigung
$g = \input{build/g.tex}$ \cite{scipy} ausgenutzt. Eine Punktmasse $m$, die sich im Abstand $\symbf r$
auf der Achse des magnetischen Moments befindet, übt dazu ein Drehmoment
\begin{equation}
	\symbf D = m \symbf r \times \symbf g
	\label{eqn:dreh_gravitation}
\end{equation}
auf den Permanentmagneten aus. Zu jeder festen Entfernung $r$ existiert jeweils genau eine Magnetfeldstärke, welche dieses
nach \eqref{eqn:dreh_magnetisch} kompensiert. So stellt sich ein Gleichgewicht
\begin{equation}
	\symbf \mu \times \symbf B = m \symbf r \times \symbf g
	\stepcounter{equation}
	\tag{8a}
	\label{eqn:dreh_gleichgewicht_vektor}
\end{equation}
ein, wobei $\symbf B$ und $\symbf g$ sowie $\symbf \mu$ und $\symbf r$ zueinander gleichgerichtet sind. Damit folgt
der Term $\mu B \sin \theta = m r g \sin \theta$ aus den Kreuzprodukten. Die Winkelabhängigkeit $\theta$ zwischen
den entsprechenden Vektoren kann gekürzt werden, sodass sich
\begin{equation}
	\mu B = m r g
	\tag{8b}
	\label{eqn:dreh_gleichgewicht_skalar}
\end{equation}
ergibt. Das magnetische Moment des betrachteten Dipols wird mit $\mu$ bemessen.

\subsubsection{Schwingung}

Über die Schwingungsdauer $T$ des Permanentmagneten kann alternativ eine dynamische Messmethode aufgestellt werden.
Formel~\eqref{eqn:dreh_magnetisch} lässt sich per Definition des Drehmoments mithilfe von Trägheitsmoment $J$ und
Winkelbeschleunigung $\ddot{\theta}$ in der Form $-\mu B \sin \theta = J \ddot{\theta}$ formulieren. Diese
Differentialgleichung wird unter Kleinwinkelnäherung zu
\begin{equation}
	-\mu B \theta = J \ddot{\theta}
	\label{eqn:harmonisch}
\end{equation}
und beschreibt damit den harmonischen Oszillator. Dessen Periodendauer
\begin{equation}
	T^2 = \pfrac{4\pi^2 J}{\mu B}
	\label{eqn:schwingung}
\end{equation}
löst die Bewegungsgleichung.

\subsubsection{Präzession}

Ein weiterer dynamischer Ansatz versetzt den Dipol in Rotation und modelliert diesen als Kreisel. Durch das Einwirken
einer äußeren Kraft beginnt die Figurenachse $\symbf L_0$ sich auf einem Kegelmantel um die Drehimpulsachse $\symbf L$
zu bewegen. Dieses Phänomen wird Präzession genannt und tritt gerade dann auf, wenn der Permanentmagnet dem Einfluss
eines homogenen Magnetfeldes ausgesetzt ist. Wieder dient hierbei ein Zusammenhang des Drehmoments dazu, um aus
\eqref{eqn:dreh_magnetisch} eine Differentialgleichung der Form
\begin{equation}
	\symbf \mu \times \symbf B =  \dot{\symbf L}_0
	\label{eqn:dreh_impuls}
\end{equation}
herzuleiten. Eine Lösung für das präzedierende System ist mit der Umlaufrate
\begin{equation}
	\Omega =  \pfrac{\mu B}{L_0}
	\label{eqn:frequenz}
\end{equation}
gegeben. Dabei lässt sich der Drehimpuls $L_0 = J \omega$ über das Trägheitsmoment $J$ und die Kreisfrequenz
$\omega = 2\pi \nu$ bestimmen. Indem die Präzessionsfrequenz $\Omega$ durch den Kehrwert der Umlaufzeit $T$
ausgedrückt wird, folgt daraus die Beziehung
\begin{equation}
	\pfrac{\hspace{0.45ex}1\hspace{0.45ex}}{\hspace{-0.15ex}T\hspace{0.15ex}} = \pfrac{\mu B}{2\pi L_0}
	\label{eqn:präzession}
\end{equation}
unter der Voraussetzung, dass die Rotationsfrequenz $\nu$ konstant ist.

\subsection{Fehlerrechnung}

Um die Abweichung der Messgrößen zu untersuchen, werden noch einige weitere Formeln benötigt.
Die Rechenvorschrift für den Mittelwert $\overline{x}$ ist mit
\begin{equation}
	\overline{x} = \pfrac{1}{N \,} \sum_{n=1}^N x_n
	\label{eqn:mittel}
\end{equation}
gegeben. Zur Bestimmung der Standardabweichung $\symup{\Delta}\overline{x}$ kann
\begin{equation}
	(\symup{\Delta}\overline{x})^2 = \pfrac{1}{N(N-1)} \sum_{n=1}^N (x_n \! - \overline{x})^2
	\label{eqn:std}
\end{equation}
verwendet werden. Durch die Gaußsche Fehlerfortpflanzung
\begin{equation}
	(\symup{\Delta}f)^2 = \sum_{n=1}^N
	\left( \! \pfrac{\partial^{\!} f}{\partial x_{\raisebox{0.2ex}{$\scriptstyle{n}$}}} \!
	\right)^{\!\! 2} \!\! (\symup{\Delta}x_{\raisebox{0.2ex}{$\scriptstyle{n}$}})^2
	\label{eqn:gauss}
\end{equation}
ist die Abweichung $\symup{\Delta}f$ für von fehlerbehafteten Werten $x_n \!$ abhängige
Größen $\hspace{-0.2ex} f \hspace{0.2ex}$ definiert.

\subsection{Lineare Regression}

Entlang der linear abhängigen Messgrößen $(x_n,y_n)$ für $n = 1,\dotsc,\hspace{-0.1ex}N\hspace{0.1ex}$
lassen sich über
\begin{align}
	a = \pfrac{\overline{x\hspace{0.15ex}y\hspace{0.15ex}} - \overline{x} \: \overline{y\hspace{0.15ex}}}
	{\overline{x^2} - \overline{x}^{\hspace{0.15ex}2}} && b = \overline{y\hspace{0.15ex}} - a \overline{x}
	\label{eqn:regression}
\end{align}
die Parameter der Ausgleichsgerade $y = ax + b$ berechnen.

