% Messwerte: Alle gemessenen Größen tabellarisch darstellen
% Auswertung: Berechnung geforderter Ergebnisse mit Schritten/Fehlerformeln/Erläuterung/Grafik (Programme)
\section{Auswertung}
\label{sec:auswertung}

Beide Reservoire sind jeweils mit einer Wassermenge von \qty{3}{\liter} gefüllt. Am Versuchsaufbau ist
außerdem die Wärmekapazität von Kupferschlange und Behälter als
$m_k c_k = \qty[per-mode=reciprocal]{750}{\joule\per\kelvin}$
abzulesen. Bei der Aufnahme der Druckmessungen ist zu beachten, dass aufgrund der Apparatur eine Addition
von genau $\qty{1}{\bar} = \qty{100}{\kilo\pascal}$ als Korrektur \mbox{vorgenommen} wird. \mbox{Nichtlineare}
Ausgleichsrechnungen werden mit den Bibliotheken NumPy \cite{numpy} und SciPy \cite{scipy}
unter Python \cite{python} durchgeführt. Zum Erstellen von Grafiken wird \mbox{Matplotlib \cite{matplotlib}
verwendet.}

\subsection{Approximation der Temperaturverläufe}

\begin{figure}[H]
	\includegraphics{build/plot_temperature.pdf}
	\caption{Temperaturverläufe nach Tabelle \ref{tab:data} mit analytischen Näherungskurven.}
	\label{fig:temp}
\end{figure}

Entlang der in Abbildung \ref{fig:temp} dargestellten Messungen soll \mbox{eine nichtlineare}
Regression durchgeführt werden. Dazu wird angenommen, dass die Temperaturverläufe einfachen
Funktionen der Form
\begin{align}
	T = A \, t^2 + B \, t + C && \pfrac{\symup dT \hspace{0.2ex}}{\symup dt} = 2A \, t + B
	\label{eqn:polyfit}
\end{align}
folgen. Da es sich um Polynome handelt, bietet sich dabei die Funktion \verb+numpy.polyfit+ an.
Diese liefert die Koeffizienten
\begin{align*}
	A_1 &= \input{build/A_1.tex} & B_1 &= \input{build/B_1.tex} & C_1 &= \input{build/C_1.tex} \\[1ex]
	A_2 &= \input{build/A_2.tex} & B_2 &= \input{build/B_2.tex} & C_2 &= \input{build/C_2.tex}
\end{align*}
zur minimalen Abweichung zwischen Ausgleichskurve und Messreihe. Auch die Fehler zum Modellverlauf
lassen sich aus der Kovarianzmatrix direkt als Quadratwurzeln ihrer Diagonalelemente berechnen. Nach
der Fehlerfortpflanzung~\eqref{eqn:gauss} ergibt sich noch
\begin{equation}
	\symup{\Delta} \! \left( \pfrac{\symup dT \hspace{0.2ex}}{\symup dt} \right) = 
	\sqrt{4 \, t^2 \! \left(\symup{\Delta}A\right)^{2} \! + \left(\symup{\Delta}B\right)^{2}}
	\label{eqn:diff_err}
\end{equation}
als zeitabhängige Abweichung der Differentialquotienten.

\subsection{Güteziffer}

Aus den Gleichungen~\eqref{eqn:therm_5} und~\eqref{eqn:therm_6} ergibt sich der zusammengesetzte Ausdruck
\begin{equation}
	\nu_{\text{real}} = \pfrac{\hspace{0.3ex} 1 \hspace{0.3ex}}{\!P\,} \,
	( m_1 c_w + m_k c_k ) \, \pfrac{\symup dT_1}{\symup dt}
	\label{eqn:güte}
\end{equation}
zur Berechnung der realen Güteziffer. Die spezifische Wärmekapazität von Wasser beträgt
$c_w = \qty[per-mode=reciprocal]{4.18}{\kilo\joule\per\kilo\gram\per\kelvin}$ \cite{h2_o}
und liefert dann $m_1 c_w = \qty[per-mode=reciprocal]{12.54}{\kilo\joule\per\kelvin}$
als einzusetzenden Wert. Mit $m_k c_k = \qty[per-mode=reciprocal]{0.75}{\kilo\joule\per\kelvin}$
ist so $m_1 c_w + m_k c_k = \qty[per-mode=reciprocal]{13.29}{\kilo\joule\per\kelvin}$ gegeben. Die
ideale Güteziffer wird nach \eqref{eqn:therm_4a} bestimmt. Um $\nu_{\text{real}}$
und $\nu_{\text{ideal}}$ zu berechnen, werden die Werte aus Tabelle~\ref{tab:data} herangezogen.
Die Differentialquotienten \eqref{eqn:polyfit} sind nach \eqref{eqn:diff_err} mit Fehlern behaftet.
In Tabelle \ref{tab:güte} werden sie in \unit[per-mode=reciprocal]{\milli\kelvin\per\second} zu den
Zeitpunkten $t$ in \unit{\second} aufgezählt.
\\
\begin{table}
	\centering
	\caption{Differentialquotienten mit Vergleich von realer und idealer Güteziffer.}
	\begin{tabular}
		{S[table-format=4.0]
		 S[table-format=2.2]
		 @{${}\pm{}$}
		 S[table-format=1.2]
		 S[table-format=1.2]
		 @{${}\pm{}$}
		 S[table-format=1.2]
		 S[table-format=2.2]
		 c}
		\toprule
		{$t$} & \multicolumn{2}{c}{$\symup dT_1/\symup dt$} &
		\multicolumn{2}{c}{$\nu_{\text{real}}$} & {$\nu_{\text{ideal}}$} &
		{$\nu_{\text{real}}/\nu_{\text{ideal}}$}\\
		\midrule
		 300 & 23.18 & 0.39 & 2.46 & 0.04 & 24.24 & \qty{10.15}{\percent} \\
		 600 & 19.58 & 0.46 & 2.10 & 0.05 & 12.80 & \qty{16.41}{\percent} \\
		 900 & 15.99 & 0.56 & 1.85 & 0.06 &  9.30 & \qty{19.89}{\percent} \\
		1200 & 12.39 & 0.68 & 1.48 & 0.08 &  7.62 & \qty{19.42}{\percent} \\
		\bottomrule
	\end{tabular}
	\label{tab:güte}
\end{table}

\subsection{Massendurchsatz}

Wie zuvor in \eqref{eqn:güte} können die Beziehungen~\eqref{eqn:therm_7} und~\eqref{eqn:therm_8}
genutzt werden, um die Formel
\begin{equation}
	\pfrac{\symup dm}{\symup dt} =
	\pfrac{\hspace{0.3ex} 1 \hspace{0.3ex}}{L\,} \,\pfrac{\symup dQ_2}{\symup dt} = 
	\pfrac{\hspace{0.3ex} 1 \hspace{0.3ex}}{L\,} \,
	( m_2 c_w + m_k c_k ) \, \pfrac{\symup dT_2}{\symup dt}
	\label{eqn:masse}
\end{equation}
herzuleiten. Um daraus den Massendurchsatz zu berechnen, wird im nachfolgenden Abschnitt die
lokal konstante Verdampfungsenthalpie $L = \input{build/L.tex}$ bestimmt. Durch Division mit
der molaren Masse $M = \qty[per-mode=reciprocal]{120.91}{\gram\per\mole}$ \cite{c_cl2_f2} von
$\ce{CCl2F2}$ ergibt sich $L = \qty[per-mode=reciprocal]{171.00(2.89)}{\joule\per\gram}$. Mit dem
Faktor $m_2 c_w + m_k c_k = \qty[per-mode=reciprocal]{13.29}{\kilo\joule\per\kelvin}$ lässt sich
nun $\symup dm/\symup dt$ ermitteln. Die fortgeführte Abweichung von \eqref{eqn:masse} lautet hierbei
\begin{equation}
	\symup{\Delta} \! \left( \pfrac{\symup dm}{\symup dt} \right) = 
	\sqrt{\left(\!\symup{\Delta}\!\left( \pfrac{\symup dT \hspace{0.2ex}}{\symup dt}\right)\!\!\right)
	^{\!\!\!2} \! L^{-2} + \left(\!\!\left( \pfrac{\symup dT \hspace{0.2ex}}{\symup dt}\right)\!
	\symup\Delta L \!\right)^{\!\!\!2} \! L^{-4} \:}
	\label{eqn:masse_err}
\end{equation}
und entspricht damit der Rechenvorschrift \eqref{eqn:gauss} nach Gauß. Zusammen mit den passenden
Differentialquotienten in \unit[per-mode=reciprocal]{\milli\kelvin\per\second} kann der fehlerbehaftete
Massendurchsatz in \unit[per-mode=reciprocal]{\gram\per\second} für die Zeitpunkte $t$ in \unit{\second}
aus Tabelle \ref{tab:masse} entnommen werden.
\begin{table}
	\centering
	\caption{Differentialquotienten mit Massendurchsatz.}
	\begin{tabular}
		{S[table-format=4.0]
		 S[table-format=4.2]
		 @{${}\pm{}$}
		 S[table-format=1.2]
		 S[table-format=3.2]
		 @{${}\pm{}$}
		 S[table-format=1.2]}
		\toprule
		{$t$} & \multicolumn{2}{c}{$\:\:\symup dT_2/\symup dt$} &
		\multicolumn{2}{c}{$\:\symup dm/\symup dt$} \\
		\midrule
		 300 & -18.17 & 0.42 & -1.41 & 0.04 \\
		 600 & -15.78 & 0.50 & -1.23 & 0.04 \\
		 900 & -13.40 & 0.61 & -1.04 & 0.05 \\
		1200 & -11.01 & 0.74 & -0.86 & 0.06 \\
		\bottomrule
	\end{tabular}
	\label{tab:masse}
\end{table}

\subsection{Verdampfungsenthalpie}

Zur Berechnung der Regressionskurve in Abbildung \ref{fig:dampf} wird das letzte Wertepaar $T_1, p_b$
aus Tabelle \ref{tab:data} exkludiert, da es nach Ausschalten des Kompressors zu einem instantanen
Sprung im Druck $p_b$ kommt. Diese Wirkung tritt in der kontinuierlichen Temperaturmessung $T_1$
erst verzögert auf und verfälscht dadurch die untersuchte Beziehung.

\begin{figure}[H]
	\includegraphics{build/plot_pressure.pdf}
	\caption{Dampfdruckkurve für \ce{CCl2F2} mit $T_1, p_b$ als Fit\hspace{0.15ex}-Daten.}
	\label{fig:dampf}
\end{figure}

Um die Dampfdruckkurve weit unterhalb der kritischen Temperatur zu beschreiben, wird hier
in guter Näherung der Ausdruck
\begin{equation}
	p = p_0 \exp\!\left( -\pfrac{L}{\symup{R}} \pfrac{\,1\,}{\!T} \right)
	\label{eqn:fit}
\end{equation}
als vereinfachtes Integrationsergebnis aus der Clausius-Clapeyron-Gleichung \cite{wärme} \mbox{verwendet.}

Dabei ist die molare Gaskonstante mit $\symup{R} = \input{build/R.tex}$ \cite{phys_const} angegeben.
Eine passende nichtlineare Ausgleichsrechnung mittels \verb+scipy.optimize.curve_fit+ liefert
\begin{align*}
	p_0 = \input{build/p.tex} && \pfrac{L}{\symup{R}} = \input{build/L_R.tex}
\end{align*}
als optimale Parameter zur Minimierung der Fehlerquadrate mit numerisch genäherter
\mbox{modellspezifischer Standardabweichung.} Daraus ergibt sich die Verdampfungsenthalpie zu
$L = \input{build/L.tex}$ für das Transportgas \mbox{Dichlordifluormethan \hspace{-0.15ex}($\ce{CCl2F2}$).}

\subsection{Mechanische Kompressorleistung}

Bevor der Ausdruck \eqref{eqn:therm_12} zur Berechnung der mechanischen Leistung $N$ verwendet werden kann,
wird eine Beziehung für die Dichte $\rho$ gesucht. Mit $m = \rho \hspace{0.2ex} V$ erfolgt dies über
\begin{equation}
	pV = \pfrac{p\hspace{0.1ex}m}{\raisebox{0.567ex}{$\! \rho$}} = nRT \iff \pfrac{p}{\rho T\hspace{0.5ex}} =
	\pfrac{nR}{\raisebox{0.567ex}{$m$}}
	\label{eqn:idealgas}
\end{equation}
als Definition der idealen Gasgleichung. Unter der Annahme, dass $nR$ sowie $m$ erhalten
bleiben, muss die Gleichheit
\begin{equation}
	\pfrac{p_0}{\rho_0 T_0} = \pfrac{p}{\rho T\hspace{0.5ex}}
	\label{eqn:gleich}
\end{equation}
für beliebige Zeitpunkte $t$ erfüllt sein. Damit können schließlich $T_2 = T$ und $p_a = p$ als entsprechende
veränderliche Messgrößen substituiert werden. Auflösen liefert
\begin{equation}
	\rho = \pfrac{\rho_0 T_0 \hspace{0.2ex} p_a}{p_0 T_2}
	\label{eqn:druck}
\end{equation}
für die auftretende Dichte. Der Term wird nun eingesetzt, um aus \eqref{eqn:therm_12} die Vorschrift
\begin{equation}
	N = \pfrac{1}{\kappa - 1} \!\left( \! p_b \sqrt[\raisebox{-1.2ex}{$\hspace{-0.4ex}^\kappa$}]
	{\pfrac{p_a}{\raisebox{0.7ex}{\(p_b\)}}}-p_a \! \right) \!
	\pfrac{p_0 T_2}{\rho_0 T_0 \hspace{0.2ex} p_a} \, \pfrac{\symup dm}{\symup dt}
	\label{eqn:leistung}
\end{equation}
zu bilden. Mit den Normalgrößen $p_0 = \qty{1}{\bar} = \qty{100}{\kilo\pascal}$ und
$T_0 = \qty{0}{\celsius} = \qty{273.15}{\kelvin}$ sowie dem stoffspezifischen Literaturwert
$\rho_0 = \qty[per-mode=reciprocal]{5510}{\gram\per\cubic\meter}$ \cite{pumpe} berechnet sich
die geforderte Kenngröße anhand der in Tabelle \ref{tab:data} geführten Messdaten. Aus Tabelle
\ref{tab:leistung} kann dazu die mechanische Leistung $N$ in \unit{\watt} zu verschiedenen Zeiten
$t$ in \unit{\second} abgelesen werden.
\\
\begin{table}
	\centering
	\caption{Mechanische Leistung des Kompressors.}
	\begin{tabular}
		{S[table-format=3.0]
		 S[table-format=3.2]
		 @{${}\pm{}$}
		 S[table-format=1.2]
		 S[table-format=2.2]
		 S[table-format=4.0]
		 S[table-format=3.2]
		 @{${}\pm{}$}
		 S[table-format=1.2]}
		\toprule
		{$t$} & \multicolumn{2}{c}{$\: N$} & {} &
		{$t$} & \multicolumn{2}{c}{$\: N$} \\
		\midrule
		 300 & -11.69 & 0.33 &&  900 & -20.50 & 0.99 \\
		 600 & -17.74 & 0.64 && 1200 & -19.73 & 1.36 \\
		\bottomrule
	\end{tabular}
	\label{tab:leistung}
\end{table}

\begin{table}
	\centering
	\captionsetup{width=1.1125\linewidth}
	\caption{Messdaten aus Temperatur $T$, Druck $\tilde{p}$, korrigiertem Druck $p$ und Kompressorleistung
			 $\hspace{-0.1ex}P\hspace{0.1ex}$ zum Zeitpunkt $t$. Zyklische Datenaufnahme
			 erfolgt beginnend bei $t$ in der Reihenfolge $T_2, \tilde{p}_a, \tilde{p}_b, T_1, P$
			 über einen Zeitraum von \qty{20(10)}{\second}.}
	\begin{adjustbox}{center}
		\input{build/table.tex}
	\end{adjustbox}
	\label{tab:data}
\end{table}
