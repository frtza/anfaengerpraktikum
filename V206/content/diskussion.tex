% Diskussion: Resultate mit Fehler/Genauigkeit zusammenstellen, Literaturwerte/Messmethoden/Ursachen vergleichen
% Literatur: Verwendete Literatur/Grafiken/Werte/Programme
% Anhang: Kopie der analog eingetragenen Messdaten
\section{Diskussion}
\label{sec:diskussion}

Alle betrachteten Kenngrößen der Wärmepumpe liegen innerhalb des Erwartungsbereichs. Die graphische
Darstellung der Näherungskurven in Abbildung \ref{fig:temp} und \ref{fig:dampf} erlaubt den Schluss,
das diese ebenfalls eine gute Deckung mit den Messdaten aufweisen. Auch die berechnete
Verdampfungsenthalpie $L = \input{build/L.tex}$ stimmt mit den Literaturwerten für ähnliche
Temperaturbereiche \cite{c_cl2_f2} überein. Auffällig ist die Abweichung der Güteziffern,
welche besonders für geringe Temperaturdifferenzen stark ansteigt. Laut Theorie \eqref{eqn:therm_4a}
könnten hier eigentlich beliebig hohe Faktoren erzielt werden. Real wirken allerdings Verluste an den
Wärmetauschern durch eventuell undichte Isolationselemente limitierend auf die Effektivität des Aufbaus.
Zusätzlich verliert die Apparatur auch durch mechanische Reibungsprozesse Energie. Abschließend darf
noch ungenaues Ablesen an den vergleichsweise groben Skalen nicht als Fehlerquelle vernachlässigt werden.

\vfill

Alle theoretischen Grundlagen werden der Versuchsanleitung \cite{pumpe} entnommen. Auch das
Schema einer Wärmepumpe ist an die darin enthaltenen Grafiken angelehnt.

\vfill
\vfill
\vfill
\vfill
\vfill
\vfill
\vfill
\vfill
\vfill
\vfill
