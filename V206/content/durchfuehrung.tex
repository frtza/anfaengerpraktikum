% Aufgabe: Messaufgaben auflisten
% Vorbereitung: Vorbereitungsaufgaben bearbeiten
% Versuchsaufbau: Verwendete Apparatur, Beschreibung Funktionsweise/Nutzen mit Skizze/Foto

\vfill

\section{Durchführung}
\label{sec:durchführung}

Die Messapparatur ist nach dem Schema in Abbildung \ref{fig:pumpe} aufgebaut. Zunächst werden die
Rührelemente aktiviert, um eine homogene Wassertemperatur in den thermisch isolierten Gefäßen
zu gewährleisten. Anschließend wird der Kompressor eingeschaltet, direkt gefolgt vom Beginn der
Messaufzeichnung. Es werden in Intervallen von \qty{60}{\second} zyklisch die Werte von Thermometern,
Manometern und Leistungsmesser abgelesen und eingetragen. Die Messreihenfolge lautet dabei
$T_2$, $p_a$, $p_b$, $T_1$, $P$. Durch die so definierte Abfolge ist eine genauere Auswertung
der Daten möglich, da die beim Nachhalten auftretende Verzögerung nur in einer vergleichsweise
geringen und annähernd konstanten zeitlichen Verschiebung der Daten zueinander resultiert.
Beendet wird der Versuch, sobald \qty{50}{\celsius} überschritten oder \qty{0}{\celsius}
unterschritten sind. Damit ist sichergestellt, dass einerseits ausreichend viele Messdaten aufgenommen
werden und es andererseits nicht zur Beschädigung des Aufbaus kommt. Die nachfolgende Analyse baut auf
den hier gewonnenen Daten auf.

\vfill

