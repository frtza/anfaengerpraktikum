% Diskussion: Resultate mit Fehler/Genauigkeit zusammenstellen, Literaturwerte/Messmethoden/Ursachen vergleichen
% Literatur: Verwendete Literatur/Grafiken/Werte/Programme
% Anhang: Kopie der analog eingetragenen Messdaten
\section{Diskussion}
\label{sec:diskussion}

Bei Betrachtung der Messergebnisse wird deutlich, dass die Werte eine gute Kompatibilität zum Modell aufweisen.
Im Vergleich mit den Literaturwerten treten allerdings signifikate Abweichungen auf. Neben der Ableseungenauigkeit
bei der Zeitmessung am Fallrohr des Viskosimeters existieren weitere mögliche Fehlerquellen, welche eher für die
um etwa ein Drittel größere ermittelte Viskosität verantwortlich sein können. So fallen beim Vermessen der Kugeln
schnell einige Sprünge an deren Oberfläche auf, die reale Verfälschungen in Volumen und Geometrie produzieren. Trotz
gleicher Materialien besitzt die große Kugel außerdem eine geringere Dichte als die kleinere. Diese Differenz dürfte
nicht allein auf die fehlenden Splitter zurückzuführen sein, lässt sich aber womöglich durch Unreinheiten innerhalb
der Objekte erklären. Eine weitere Möglichkeit, Fehler in die Messungen einzuführen, besteht in der geforderten
Blasenfreiheit. Diese lässt sich nur visuell prüfen. Schon sehr feine Restbläschen können drastischen Einfluss auf
das Resultat der Messungen haben. Auch beim Drehen der Apparatur könnten Variationen auftreten, da die Raste nicht
immer greift. Eine letzte plausible Begründung stellt das kontinuierliche Erhitzen am Thermostat dar. Durch eine recht
grobe Einstellung ist nicht garantiert, dass die momentane mit der nachgehaltenen Temperatur übereinstimmt. Dies sollte
allerdings nur zu leichten statistischen Ausreißern führen und im Mittel wegfallen. Turbulenzen können wegen der niedrigen
Reynoldsschen Zahlen weitestgehend ausgeschlossen werden.

\newpage
