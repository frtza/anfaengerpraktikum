% Messwerte: Alle gemessenen Größen tabellarisch darstellen
% Auswertung: Berechnung geforderter Ergebnisse mit Schritten/Fehlerformeln/Erläuterung/Grafik (Programme)
\section{Auswertung}
\label{sec:auswertung}

Die hier verwendeten Komponenten weisen baubedingt gewisse Toleranzbereiche sowie unsystematische Fehler auf.
Diese sind in relativer Darstellung angegeben, können für die Fehlerfortpflanzung also direkt als Verhältnis
in~\eqref{eqn:speziell} eingesetzt werden. Außerdem folgt aufgrund der Funktionsweise des Potentiometers
für Versuche, in denen ein solches zur Einstellung eines Widerstandsverhältnisses gebraucht wird:
\begin{equation*}
	\pfrac{\vphantom{\int}R_3}{\vphantom{\int}R_4} =
	\pfrac{\vphantom{\int}R_3}{\vphantom{\int} \qty{1}{\kilo\ohm} - \! R_3}
\end{equation*}
Zur Bestimmung der Abweichung wird jeweils die Standardabweichung des Mittelwertes nach~\eqref{eqn:std}
gebildet und mit dem Ergebnis der Fehlerfortpflanzung~\eqref{eqn:gauss} verglichen. Der größere Wert
wird dann als Unsicherheit festgelegt. Zur graphischen Darstellung sowie zum automatisierten Ausgeben
der Tabellen werden die Bibliotheken NumPy \cite{numpy} und Matplotlib \cite{matplotlib}
unter Python \cite{python} genutzt.

\subsection{Wheatstonesche Messbrücke}

\begin{table}
	\centering
	\caption{Messdaten zur Bestimmung ohmscher Widerstände unter Anwendung der
			 Wheatstoneschen Brücke bei $R_4 = \qty{1}{\kilo\ohm} - R_3$.}
		\begin{tabular}
		{S[table-format=4.0]
		 S[table-format=3.0]
		 S[table-format=3.3] 
		 S[table-format=3.0]
		 S[table-format=3.3]}
		\toprule
		& \multicolumn{2}{c}{Wert 13} & \multicolumn{2}{c}{Wert 14} \\
		\cmidrule(lr){2-3} \cmidrule(lr){4-5}
		{$R_2 \mathbin{/} \unit{\ohm}$} &
		{$R_3 \mathbin{/} \unit{\ohm}$} &
		{$R_x \mathbin{/} \unit{\ohm}$} &
		{$R_3 \mathbin{/} \unit{\ohm}$} &
		{$R_x \mathbin{/} \unit{\ohm}$} \\
		\midrule
		 500 & 391 & 321.018 & 645 & 908.451 \\
		 664 & 326 & 321.163 & 577 & 905.740 \\
		1000 & 243 & 321.004 & 475 & 904.762 \\
		\bottomrule
	\end{tabular}

	\label{tab:ohm}
\end{table}

Die relative Abweichung für $R_2$ ist mit $\qty{0.2}{\percent}$ angegeben, das am Potentiometer erzeugte
Verhältnis von $R_3$ zu $R_4$ besitzt mit $\qty{0.5}{\percent}$ eine etwas größere Ungenauigkeit. $R_x$
berechnet sich nach Formel~\eqref{eqn:wheatstone}. Daraus und aus den in Tabelle \ref{tab:ohm}
nachgehaltenen Messergebnissen ergeben sich die Mittelwerte mit ihren baubedingten absoluten Fehlern: 
\begin{align*}
	R_{x,13} = \qty{321.1(1.7)}{\ohm} &&
	R_{x,14} = \qty{906.3(4.9)}{\ohm}
\end{align*}
Die Abweichungen der Mittelwerte fallen mit $\symup{\Delta}R_{x,13} = \qty{0.1}{\ohm}$ 
und $\symup{\Delta}R_{x,14} = \qty{1.1}{\ohm}$ deutlich geringer aus. \newpage

\subsection{Kapazitätsmessbrücke}

\begin{table}
	\centering
	\caption{Messdaten zur Bestimmung von Kapazität und Verlustwiderstand.}
		\begin{tabular}
		{S[table-format=3.0]
		 S[table-format=3.0]
		 S[table-format=3.0]
		 S[table-format=3.3]
		 S[table-format=3.3]}
		\toprule
		{$C_2 \mathbin{/} \unit{\nano\farad}$} &
		{$R_2 \mathbin{/} \unit{\ohm}$} &
		{$R_3 \mathbin{/} \unit{\ohm}$} &
		{$C_x \mathbin{/} \unit{\nano\farad}$} &
		{$R_x \mathbin{/} \unit{\ohm}$} \\
		\midrule \\[-1.5ex]
		& \multicolumn{4}{c}{Wert 8} \\
		\cmidrule(lr){2-5}
		450 & 380 & 606 &  292.574 & 584.467 \\
		597 & 288 & 671 &  292.717 & 587.380 \\[1.5ex]
		& \multicolumn{4}{c}{Wert 3 mit Wert 13} \\
		\cmidrule(lr){2-5}
		450 & 412 & 484 &  479.752 & 386.450 \\
		597 & 232 & 589 &  416.582 & 332.477 \\[1.5ex]
		& \multicolumn{4}{c}{Wert 4 mit Wert 13} \\
		\cmidrule(lr){2-5}
		450 & 694 & 321 &  951.869 & 328.091 \\
		597 & 615 & 350 & 1108.714 & 331.154 \\
		\bottomrule
	\end{tabular}

	\label{tab:kapazität}
\end{table}

Der relative Fehler von $C_2$ beträgt $\qty{0.2}{\percent}$, während das Verhältnis von $R_3$ zu $R_4$
weiterhin mit einer Abweichung von $\qty{0.5}{\percent}$ behaftet ist. Um den verstellbaren Widerstand
$R_2$ zu realisieren, wird ein einzelner Ausgang eines Potentiometers genutzt. Der zugehörige Fehler ist
mit $\qty{3}{\percent}$ gegeben. Zur Messung der Kapazitäten werden sie mit Wert 13 {in} Reihe geschaltet.
Dies ermöglicht es, die Verlustwiderstände mit den vorherigen Ergebnissen zu vergleichen. Unter Verwendung
von~\eqref{eqn:kapazität} lassen sich die in Tabelle \ref{tab:kapazität} eingetragenen Messungen auswerten.

Für Wert 4 ergibt sich ein Fehler von $\qty{5.5}{\nano\farad}$, welchen die Standardabweichung
eindeutig übertrifft. Dagegen ist für den kombinierten Verlustwiderstand die Abweichung des Mittelwertes
mit $\symup{\Delta} R_{x,4} = \qty{1.5}{\ohm}$ geringer:
\begin{align*}
	C_{x,4} = \qty{1030(78)}{\nano\farad} && R_{x,4} = \qty{330(10)}{\ohm}
\end{align*}
Zu Wert 3 liegen beide baubedingten Fehler mit $\qty{2.4}{\nano\farad}$ und $\qty{11}{\ohm}$ deutlich
unter ihren entsprechenden Mittelwertsabweichungen:
\begin{align*}
	C_{x,3} = \qty{448(32)}{\nano\farad} &&
	R_{x,3} = \qty{359(27)}{\ohm}
\end{align*}
Der Vergleich der kombinierten Wirkwiderstände mit dem einzelnen zuvor gemessenen Bauteil
$R_{x,13} = \qty{321.1(1.7)}{\ohm}$ zeigt, dass die realen Kondensatoren tatsächlich geringere Verluste
aufweisen, diese aber nicht unbedingt völlig vernachlässigbar sind. 

Für die $RC$-Kombination Wert 8 sind die Messungen ähnlicher, die Standardabweichungen 
$\symup{\Delta} C_{x,8} = \qty{0.1}{\nano\farad}$ und $\symup{\Delta} R_{x,8}= \qty{1.5}{\ohm}$
fallen geringer als die baubedingten Fehler aus:
\begin{align*}
	C_{x,8} = \qty{292.6(1.6)}{\nano\farad} &&
	R_{x,8} = \qty{586(18)}{\ohm}
\end{align*}

\subsection{Induktivitätsmessbrücke}
\label{sec:4.3}

\begin{table}
	\centering
	\caption{Messdaten zur Bestimmung von Induktivität und Verlustwiderstand.}
		\begin{tabular}
		{S[table-format=2.1]
		 S[table-format=2.0]
		 S[table-format=3.0]
		 S[table-format=3.2]
		 S[table-format=3.2]}
		\toprule
		\multicolumn{5}{c}{Wert 18} \\
		\cmidrule(lr){1-5}
		{$L_2 \mathbin{/} \unit{\milli\henry}$} &
		{$R_2 \mathbin{/} \unit{\ohm}$} &
		{$R_3 \mathbin{/} \unit{\ohm}$} &
		{$L_x \mathbin{/} \unit{\milli\henry}$} &
		{$R_x \mathbin{/} \unit{\ohm}$} \\
		\midrule
		20.1 & 24 & 849 & 113.01 & 134.94 \\
		14.6 & 26 & 895 & 124.45 & 221.62 \\
		\bottomrule
	\end{tabular}

	\label{tab:induktivität}
\end{table}

Die Induktivität $L_2$ besitzt eine relative Abweichung \qty{0.2}{\percent}. Für Größen, \mbox{welche ohmsche}
Widerstände enthalten, gelten die gleichen Ungenauigkeiten wie vorher. Werden die
Ausdrücke~\eqref{eqn:induktivität} auf die Messergebnisse aus Tabelle \ref{tab:induktivität} angewendet,
überwiegt erneut die Mittelwertsabweichung im Vergleich zu den baubedingten Fehlern $\qty{0.7}{\milli\henry}$
und $\qty{5.4}{\ohm}$:
\begin{align*}
	L_{x,18} = \qty{118.7(5.7)}{\milli\henry} && R_{x,18} = \qty{178(43)}{\ohm}
\end{align*}

\subsection{Maxwell-Brücke}
\label{sec:4.4}

\begin{table}
	\centering
	\caption{Messdaten zur Bestimmung von Induktivität und Verlustwiderstand mittels der
			 Maxwell-Brücke bei $R_2 = \qty{1}{\kilo\ohm}$.}
		\begin{tabular}
		{S[table-format=3.0]
		 S[table-format=3.0]
		 S[table-format=3.0]
		 S[table-format=3.3]
		 S[table-format=3.3]}
		\toprule
		\multicolumn{5}{c}{Wert 16} \\
		\cmidrule(lr){1-5}
		{$C_4 \mathbin{/} \unit{\nano\farad}$} &
		{$R_3 \mathbin{/} \unit{\ohm}$} &
		{$R_4 \mathbin{/} \unit{\ohm}$} &
		{$L_x \mathbin{/} \unit{\milli\henry}$} &
		{$R_x \mathbin{/} \unit{\ohm}$} \\
		\midrule
		450 & 284 & 506 & 127.800 & 561.265 \\
		597 & 225 & 314 & 134.325 & 716.561 \\
		\bottomrule
	\end{tabular}

	\label{tab:maxwell}
\end{table}

Nun werden zwei variable Widerstände $R_3$ und $R_4$ eingebaut, die beide eine Toleranz von
$\qty{3}{\percent}$ besitzen. Für die Kapazität $C_4$ und den festen Widerstand $R_2$ kann wieder
von einem relativen Fehler $\qty{0.2}{\percent}$ ausgegangen werden. Mit~\eqref{eqn:maxwell} lassen
sich aus den Werten in Tabelle \ref{tab:maxwell} folgende Ergebnisse bestimmen:
\begin{align*}
	L_{x,16} = \qty{131.1(3.9)}{\milli\henry} && R_{x,16} = \qty{639(78)}{\ohm}
\end{align*}
Für die Induktivität fällt die Standardabweichung des Mittelwertes
$\symup{\Delta} L_{x,16} = \qty{3.3}{\milli\henry}$ knapp geringer aus, beim Verlustwiderstand ist wiederum
der baubedingte Fehler mit $\qty{27}{\ohm}$ klar niedriger.

\subsection{Frequenzverhalten der Wien-Robinson-Brücke}

Die in der Schaltung verbauten Komponenten haben die Werte $R' \! = \qty{500}{\ohm}$, $R = \qty{664}{\ohm}$
und $C = \qty{450}{\nano\farad}$. Damit ergibt sich folgende Sperrfrequenz: 
\begin{equation*}
	\nu_0 = \pfrac{\omega_0}{2\pi} = \pfrac{1}{2\pi RC} = \qty{532.65}{\hertz}
\end{equation*}
Unter Betrachtung von Abbildung \ref{fig:plot} und Tabelle \ref{tab:wien} wird direkt deutlich, dass bei
$\Omega = 1$ ein Extremum liegt, genau wie es nach~\eqref{eqn:wien} zu erwarten wäre. Allerdings
handelt es sich dabei um ein Maximum statt einem Minimum. Auch global scheinen die Daten eher der Übertragung
eines Bandpasses anstelle der eines Sperrfilters zu folgen. Mögliche Gründe dafür werden in Abschnitt
\ref{sec:diskussion} diskutiert. Zur Durchführung der anschließenden Rechnungen wird davon ausgegangen,
dass die Daten in guter Näherung einer verzerrten Spiegelung der Theoriekurve entsprechen.

\begin{figure}[H]
	\centering
	\includegraphics{build/plot.pdf}
	\caption{Messdaten mit Theoriekurve zum Frequenzverhalten der Wien-Robinson-Brücke
			 in linearer und halblogarithmischer Skalierung.}
	\label{fig:plot}
\end{figure}

\begin{table}[H]
	\centering
	\caption{Messdaten zur Untersuchung des Frequenzverhaltens einer Wien-Robinson-Brücke mit
			 $R' \! = \qty{500}{\ohm}, R = \qty{664}{\ohm}$ und $C = \qty{450}{\nano\farad}$.}
	\input{build/table_e.tex}
	\label{tab:wien}
\end{table}

\subsection{Klirrfaktor des Generators}

Um den Klirrfaktor nach~\eqref{eqn:klirr} zu bestimmen, wird zunächst in erster Approximation angenommen,
dass sich die Summe der Oberwellen nur aus der ersten Oberschwingung $U_{\! 1}$ mit $\nu = 2 \, \nu_0$
zusammensetzt. Als übrig bleibende Brückenspannung wird die Differenz aus der gemessenen maximalen
Übertragung $\num{0.42}$ und dem asymptotischen Verhalten der Theoriekurve $\num{0.33}$ gebildet. Mit
einer Speisespannung $\symup{U_{\! S}} = \qty{1}{\volt}$ ergibt sich somit
$\symup{U_{\! Br}} = \qty{0.09}{\volt}$. Dabei ist anzumerken, dass es sich weiterhin immer um die
Amplituden der Schwingungen handelt. Das Verhältnis von $\symup{U_{\! Br}}$ und $U_{\! 1}$
ergibt sich durch Einsetzen von $\Omega = 2$ in~\eqref{eqn:wien} zu \num{0.15}. Nun kann berechnet
werden:
\begin{equation*}
	U_{\! 1} = \pfrac{\qty{0.09}{\volt}}{\num{0.15}} = \qty{0.6}{\volt}
\end{equation*}
Da die Grundschwingung $U_{\! 0} = \qty{1}{\volt}$ entspricht, würde sich für den Generator damit ein
Klirrfaktor von $k = \num{0.6}$ ergeben. 

Um dieses Ergebnis unter Berücksichtigung des Gesamtverlaufs der Aufzeichnungen zu prüfen, werden weitere
Überlegungen angestellt. Statt einer direkten korrektiven Spiegelung kann vereinfacht angenommen werden,
dass sich die tatsächliche Übertragung als Superposition aus den Übertragungsfunktionen von Grundschwingung
und erster Oberschwingung darstellen lässt. Der Einfluss der letzteren ist dabei um den Klirrfaktor
$k$ abgeschwächt. Aus~\eqref{eqn:wien} bildet sich damit folgender Ausdruck:
\begin{align*}
	\qquad F_0 &= \pfrac{\, 1 \, }{\raisebox{0.5ex}{$3$}} \! \left(
	\!\! \left( \! \pfrac{\left( \Omega^2 - 1 \right)^{\! 2}}
	{\left( 1 - \Omega^2 \right)^{\! 2} \! + 9 \Omega^2} \! \right)
	^{\!\! \frac{\raisebox{0ex}{$\scriptstyle1$}}{\raisebox{-0.3ex}{$\scriptstyle2$}}} \!\! + \, k \!
	\left( \! \pfrac{\left( 4 \Omega^2 - 1 \right)^{\! 2}}
	{\left( 1 - 4 \Omega^2 \right)^{\! 2} \! + 36 \Omega^2} \! \right)
	^{\!\! \frac{\raisebox{-0ex}{$\scriptstyle1$}}{\raisebox{-0.3ex}{$\scriptstyle2$}}} \right) \\
\intertext{Die Spiegelung $F_1$ der Gleichung $F_0$ soll für $\Omega = 0$ immer Null sein:}
	\qquad F_1 &= \pfrac{\, 1 \, }{\raisebox{0.5ex}{$3$}} \! \left(
	\! 1 + k - \! \left( \! \pfrac{\left( \Omega^2 - 1 \right)^{\! 2}}
	{\left( 1 - \Omega^2 \right)^{\! 2} \! + 9 \Omega^2} \! \right)
	^{\!\! \frac{\raisebox{-0ex}{$\scriptstyle1$}}{\raisebox{-0.3ex}{$\scriptstyle2$}}} \!\! - \, k \!
	\left( \! \pfrac{\left( 4 \Omega^2 - 1 \right)^{\! 2}}
	{\left( 1 - 4 \Omega^2 \right)^{\! 2} \! + 36 \Omega^2} \! \right)
	^{\!\! \frac{\raisebox{-0ex}{$\scriptstyle1$}}{\raisebox{-0.3ex}{$\scriptstyle2$}}} \right)
\end{align*}
Es wird mithilfe von SciPy \cite{scipy} unter Python \cite{python} eine nichtlineare Regression entlang
$F_1$ auf die Daten in Tabelle \ref{tab:wien} angewendet. Dieses Vorgehen liefert mit
$k = \input{build/k.tex}$ einen ähnlichen Klirrfaktor wie zuvor. \vspace{-0.35ex}
\begin{figure}[H]
	\centering
	\includegraphics{build/fit.pdf}
	\caption{Messdaten und gespiegelte Kurven $F \,$ nach Anwendung der nichtlinearen
			 Näherung kleinster Abweichungsquadrate.}
	\label{fig:fit}
\end{figure}
