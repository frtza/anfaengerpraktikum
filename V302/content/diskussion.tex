% Diskussion: Resultate mit Fehler/Genauigkeit zusammenstellen, Literaturwerte/Messmethoden/Ursachen vergleichen
% Literatur: Verwendete Literatur/Grafiken/Werte/Programme
% Anhang: Kopie der analog eingetragenen Messdaten
\section{Diskussion}
\label{sec:diskussion}

\subsection{Ausmessen elektronischer Bauteile}

Für die Messungen von Widerstand, Kapazität und Induktivität ist durch die rein baubedingten Fehler
allgemein davon auszugehen, dass abgeleitete Ergebnisse eine hohe Genauigkeit aufweisen. Bei der
Betrachtung der in Abschnitt \ref{sec:auswertung} berechneten Werte fallen jedoch auch einige Ausreißer
mit vergleichsweise großen Abweichungen auf. Diese lassen sich teilweise durch unbekannte Größen
der Referenzbauteile begründen. Besonders für Induktivitäten lassen sich innere Verlustwiderstände nie
ganz vermeiden. Kapazitäten sind davon weniger betroffen. Auch Fluktuationen von Frequenz und Amplitude
der Generatorspannung können zu schlecht reproduzierbaren Ergebnissen führen. Nicht zu vernachlässigen
ist außerdem, dass durch menschliches Irrtum etwa bei der Einstellung der Spannungsminima fehlerhafte
Werte aufgenommen werden können. All diese Faktoren sind für Messgrößen mit größerer statistischer
Streuung kompensiert, indem mehrere Messungen durchgeführt werden.

Ein gesonderter Vergleich der klassischen Induktionsmessbrücke \hyperref[sec:4.3]{4.3} mit der
Maxwell-Brücke \hyperref[sec:4.4]{4.4} kann nicht erbracht werden, da das auszumessende Bauteil
sich nach dem Umbau der Schaltungen nicht mehr abgleichen lässt. Grund dafür ist wahrscheinlich
ein unerkannter Defekt an der Steckverbindung. Erwartet wird hier ein geringerer Fehler unter der
Maxwell-Bauweise, da diese statt einer Spule einen Kondensator mit geringerem Wirkwiderstand verwendet.

\subsection{Untersuchung des Frequenzverhaltens}

Die Verarbeitung der Messergebnisse zur Wien-Robinson-Brücke gestaltet sich stark problematisch. Da die
Werte genau gegensätzlich zur theoretischen Vorhersage laufen, müssen grobe Annahmen getroffen
werden, um überhaupt eine Analyse zu ermöglichen. Der resultierende Klirrfaktor im Bereich von
$k = \num{0.6}$ wird damit unerwartet groß. Ein solcher Oberschwingungsanteil von $\qty{60}{\percent}$ kann
zudem durch die Erkenntnisse aus den bereits erfolgten Messungen ausgeschlossen werden. Es gilt also,
mögliche signifikante Fehlerquellen zu bestimmen. Als erster Anhaltspunkt werden in der Brückenschaltung
abweichende Kapazitäten verbaut: Da keine identischen Bauteile zur Verfügung stehen, muss einmal
$\qty{450}{\nano\farad}$ durch $\qty{420}{\nano\farad}$ ersetzt werden. Diese Ungenauigkeit
kann aber nur das Minimum der Übertragungsfunktion leicht verschieben und ist damit nicht für
die Umkehrung der Erwartungswerte verantwortlich. Weiter ist es möglich, dass eine fehlerhafte
Einstellung am Oszilloskop die Messdaten verfälscht. Da dieses sowohl zur Einstellung der
Speisepannung auf $\qty{1}{\volt}$ als auch zum Ablesen der Spannungsverhältnisse dient, lassen
sich durch ungenaues Kalibrieren der Auflösung $\unit{\volt} \! / \symup{div}$ tatsächlich
stärkere Abweichungen in der Form der Messverteilung erklären. Allerdings sollte es sich auch
dabei nur um Streckungen und Stauchungen handeln, nicht aber um ein vollständiges Umkehren.
Für dieses offensichtliche Mängel bleiben noch zwei Ursachen offen: \newline Am Oszillograph kann eine
Erdung gewählt werden, die den beobachteten Effekt eines gespiegelten Betrags hervorrufen würde.
Diese Einstellungsänderung müsste aber auch die vorherigen Messungen beeinflussen und ist demnach
eher unwahrscheinlich. Als letzte Möglichkeit kann ein Fehler im Schaltungsaufbau aufgetreten sein.
Dann würde womöglich das Bandpass-Glied innerhalb der Wien-Robinson-Brücke ausgemessen, dessen
Übertragungsfunktion nicht nur gespiegelt ist, sondern auch eine andere Form aufweist. Dass dieses Problem
in mehrfach unabhängig aufgebauten und geprüften Versuchen wiederholt auftritt, scheint aber ebenfalls
nicht plausibel. Der Ursprung der verkehrten Messreihe bleibt also zunächst ungeklärt. 

\vfill

Alle Informationen sind aus der Versuchsanleitung \cite{brücke} entnommen. Auch die Schaltbilder
sind an die darin enthaltenen Grafiken angelehnt.

\vfill
\vfill
\vfill
\vfill
