% Aufgabe: Messaufgaben auflisten
% Vorbereitung: Vorbereitungsaufgaben bearbeiten
% Versuchsaufbau: Verwendete Apparatur, Beschreibung Funktionsweise/Nutzen mit Skizze/Foto
\section{Aufgabe}

Ziel dieser Versuchsreihe ist es, verschiedene elektronische Bauteile auszumessen. Dazu werden unbekannte
ohmsche Widerstände, Kapazitäten und Induktivitäten in eine jeweils passende Brückenschaltung eingebaut.
Weiter soll das frequenzabhängige Verhalten einer Filter-Brücke untersucht und mit ihrer Hilfe der
Klirrfaktor eines Sinusgenerators bestimmt werden.


\section{Durchführung}
\label{sec:durchführung}

Alle Schaltungen werden mit einer sinusförmigen Wechselspannung gespeist. Um die Komponenten zu
schonen, wird ihre Amplitude auf $\symup{U_{\! S}} = \qty{1}{\volt}$ beschränkt. Während der Messung
elektronischer Bauteile ist die Frequenz konstant auf $\nu = \qty{1}{\kilo\hertz}$ festgelegt, da so
Wirk- und Blindwiderstände im Bereich einer Größenordnung liegen. Dadurch lassen sich unerwünschte
Streukapazitäten und lange Einschwingvorgänge minimieren. Für Widerstandsverhältnisse und variable
Widerstände wird ein Drehregler-Potentiometer mit Gesamtwiderstand $R = \qty{1}{\kilo\ohm}$ verbaut.
Als Messgerät für die Brückenspannung dient ein Kathodenstrahloszillograph. Dieser ist mit
$\qty{5}{\milli\volt} \! / \symup{div}$ auf die feinste Einstellung kalibriert.

\subsection{Widerstandsmessung}

Zunächst wird eine Wheatstonesche Brücke gemäß Abbildung \ref{fig:wheatstone} aufgebaut. Das Abgleichen
der Schaltung erfolgt durch Variation des Potentiometers. Anschließend werden die Werte $R_2$ und $R_3$
notiert. Da für das Widerstandsverhältnis $R_4 = \qty{1}{\kilo\ohm} - \! R_3$ gilt, muss $R_4$ nicht explizit
nachgehalten werden. Dieses Vorgehen wird für zwei unbekannte Werte mit jeweils drei verschiedenen $R_2$
wiederholt.

\subsection{Kapazitätsmessung}

Die Kapazitätsmessbrücke folgt im Aufbau dem Schaltplan in Abbildung \ref{fig:kapazität}. Gemessen
werden zwei unbekannte Kondensatoren und eine $RC$-Kombination. Obwohl bei den Kapazitäten davon auszugehen
ist, dass sie nur geringe Verlustwiderstände besitzen, wird ein zuvor berechneter ohmscher Widerstand in Reihe
geschaltet. Damit wird für alle Messungen das verstellbare Kompensationsglied $R_2$ benötigt. Durch abwechselndes
Anpassen des Verhältnisses $R_3$ zu $R_4$ und von $R_2$ lassen sich Betrag und Phase der Brückenspannung
minimieren. Nachdem die Werte $C_2$, $R_2$ und $R_3$ notiert sind, wird $C_2$ jeweils einmal ausgetauscht
und das Verfahren erneut durchgeführt.

\subsection{Induktivitätsmessung}

Es werden verlustbehaftete Induktivitäten gemessen, deren innerer Wirkwiderstand nicht vernachlässigt werden
kann. Demnach müssen wieder zwei Freiheitsgrade verbaut sein. Zunächst nutzt man einen Aufbau nach Abbildung
\ref{fig:induktivität}. Durch Variieren von Potentiometer und $R_2$ wird die Schaltung so eingestellt, dass
die Brückenspannung verschwindet. Die Werte $L_2$, $R_2$ und $R_3$ werden für zwei bekannte $L_2$ notiert.

Nun wird die vorherige Induktivitätsmessbrücke durch eine Maxwell-Brücke ersetzt, deren Aufbau Abbildung
\ref{fig:maxwell} entnommen werden kann. Es wird ein gleichbleibender Widerstand $R_2$ verbaut, allerdings
werden $R_3$ und $R_4$ jetzt unabhängig voneinander angepasst. Für die abgeglichene Brücke werden $C_4$,
$R_3$ und $R_4$ nachgehalten. Die Messung wird für zwei verschiedene $C_4$ wiederholt.

\subsection{Untersuchung frequenzabhängiger Brückenschaltungen}

Eine Wien-Robinson-Brücke wird gemäß Abbildung \ref{fig:wien} aufgebaut. Hierbei ist das Ziel die Analyse des
Frequenzverhaltens der Brückenspannung. Entsprechend kann die Frequenz nicht mehr konstant bleiben, sondern
wird von $\qty{20}{\hertz}$ bis $\qty{30}{\kilo\hertz}$ variiert. Dabei werden dreißig Datentupel aus $\nu$ und
der dabei auftretenden Amplitude $\symup{U_{\! Br}}$ gebildet. Eine gleichbleibende Einstellung der
Generatorspannung $\symup{U_{\! S}}$ ist Voraussetzung für dieses Vorgehen. Die so aufgenommenen Werte werden
im Anschluss noch zur Klirrfaktorbestimmung herangezogen.
