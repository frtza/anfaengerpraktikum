% Titel: Ziel des Versuchs
% Theorie: Physikalische Grundlagen von Versuch/Messverfahren, Gleichungen ohne Herleitung knapp erklären
\section{Theorie}
\label{sec:theorie}


\subsection{Grundlegendes zur Brückenschaltung}

\begin{figure}
	\centering
	\begin{circuitikz}
		\usetikzlibrary{arrows.meta}

\tikzstyle{every node} = [font = \small]

\ctikzset{bipoles/thickness = 1}
\ctikzset{bipoles/length = 1cm}

\begin{scope}[line width = 1pt]
	\draw
		(3,-2.25) to[short, *-] ++(0,-0.75)
		\pgfextra{\ctikzset{bipoles/length = 1.4cm}}
		to[short] ++(-3.25,0) to[esource, name = U] ++(0,6)
		\pgfextra{\ctikzset{bipoles/length = 1cm}}
		to[short] ++(3.25,0) to[short, -*] ++(0,-0.75);
	\draw
		(3,2.25) to[short] ++(-1.5,0)
		to[R, l_=$R_1$, i>_=$I_1$] ++(0,-2.25)
		to[R, l_=$R_2$, i_=$I_2$] ++(0,-2.25)
		to[short] ++(1.5,0);
	\draw
		(3,2.25) to[short] ++(1.5,0)
		to[R, l^=$R_3$, i>^=$I_3$] ++(0,-2.25)
		to[R, l^=$R_4$, i^=$I_4$] ++(0,-2.25)
		to[short] ++(-1.5,0);
	\draw
		(1.5,0) to[short, *-*] ++(1.25,0);
	\draw
		(4.5,0) to[short, *-*] ++(-1.25,0);
	\draw[-{Triangle[angle = 45:1pt 2]}, shift={(3,1.125)}]
		(240:0.25cm) arc (240:-70:0.25cm);
	\draw[-{Triangle[angle = 45:1pt 2]}, shift={(3,-1.125)}]
		(240:0.25cm) arc (240:-70:0.25cm);
	\node
		at (2.977,0.333) {$\symup{U_{\! Br}}$};
	\node
		at (-0.25,-0.05) {$\symup{U_{\! S}}$};
\end{scope}

	\end{circuitikz}
	\caption{Prinzipielle Brückenschaltung.}
	\label{fig:bridge}
\end{figure}

Im Allgemeinen dient eine Brückenschaltung der Untersuchung einer Potentialdifferenz, welche
jeweils von den Verhältnissen der verbauten Widerstände abhängt. Sie wird als Brückenspannung
$\symup{U_{Br}}$ bezeichnet. Für ihre Berechnung aus der Speisespannung $\symup{U_{\! S}}$
werden die Kirchhoffschen Gesetze benötigt.

\subsubsection{Kirchhoffsche Gesetze}

An den Verzweigungen des Schaltkreises können sowohl zufließende Ströme $(I > 0)$ als auch
abfließende Ströme $(I<0)$ definiert werden. Die \textbf{Knotenregel} besagt dann, dass die Summe
der ein- und auslaufenden elektrischen Ströme an einem beliebigen Punkt immer gleich Null ist:
\begin{equation}
	\sum_k I_k = 0
	\label{eqn:knoten}
\end{equation}
Für die Brückenschaltung in Abbildung \ref{fig:bridge} folgt daraus direkt, dass $I_1 = I_2$ und
\mbox{$I_3 = I_4$} gelten muss. Weiter besagt die \textbf{Maschenregel}, dass die Summe aller
Spannungen in einem geschlossenen Unterkreis eines elektrischen Netzwerkes gleich Null ist.
Per Konvention ist die positive Richtung dabei als rechtsdrehend definiert. Eine bestimmte
elektromotorische Kraft entspricht demnach der Summe der übrigen Potentiale. Wird der
Spannungsabfall an einem Widerstand als dessen Produkt mit der Stromstärke geschrieben, ergibt sich:
\begin{equation}
	U = \sum_k U_{\! k} = \sum_k I_k R_k
	\label{eqn:masche}
\end{equation}
Damit kann der folgende Ausdruck für die Brückenspannung hergeleitet werden:
\begin{equation}
	\symup{U_{\! Br}} =
	\pfrac{R_2R_3-R_1R_4}{R_3+R_4} \, I_1 =
	\pfrac{R_2R_3-R_1R_4}{(R_3+R_4)(R_1+R_2)} \, \symup{U_{\! S}}
	\label{eqn:bridge}
\end{equation}

\subsubsection{Abgleichbedingung}

Verschwindet diese für beliebige Speisespannungen, wird die entsprechende Schaltung
abgeglichene Brücke genannt. Anhand von~\eqref{eqn:bridge} lässt sich ablesen, dass dies
genau dann der Fall ist, wenn die \textbf{Abgleichbedingung} erfüllt ist:
\begin{equation}
	R_1R_4 = R_2R_3
	\label{eqn:abgleich}
\end{equation}
Diese Relation erlaubt das Ausmessen eines unbekannten Widerstandes, solange die Werte der
übrigen Bauteile bekannt sind.

\subsubsection{Komplexe Widerstände}

Um~\eqref{eqn:abgleich} auch auf Kondensatoren und Spulen anwenden zu können, müssen noch
\textbf{komplexe Widerstände} eingeführt werden. Solche lassen sich allgemein mit der
imaginären Einheit $i \mkern1mu$ ausdrücken:
\begin{equation}
	Z = X + i \mkern1mu Y
	\label{eqn:komplex}
\end{equation}

Dabei bezeichnet $X$ den leistungsverbrauchenden Wirkwiderstand. Der Blindwiderstand $Y$
erzeugt eine zeitliche Phasenverschiebung zwischen Stromstärke und Spannung, ohne dass es
zur Umwandlung von Energie kommt. \newline Es werden die Widerstandsoperatoren für eine
Kapazität $C$, eine Induktivität $L$ und einen ohmschen Widerstand $R$ mit der Kreisfrequenz
$\omega = 2\pi \nu$ angegeben:
\begin{align}
	Z_C = \pfrac{1}{i \mkern1mu \omega C} && Z_L = i \mkern1mu \omega L && Z_R = R
	\label{eqn:operatoren}
\end{align}
Durch Einsetzen von $Z$ in~\eqref{eqn:abgleich} ergibt sich $Z_1Z_4 = Z_2Z_3$ als Abgleichbedingung.
Da für Gleichheit im komplexen Zahlenraum jeweils Real- und Imaginärteil übereinstimmen müssen, lässt
sich äquivalent formulieren:
\begin{equation}
	\begin{aligned}
		X_1X_4 - Y_1Y_4 &= X_2X_3 - Y_2Y_3 \\
		X_1Y_4 + X_4Y_1 &= X_2Y_3 + X_3Y_2
	\end{aligned}
	\label{eqn:wechsel}
\end{equation}
Zum Abgleichen einer Wechselstrombrücke müssen Betrag und Phase verschwinden, woraus sich die beiden
Bedingungen~\eqref{eqn:wechsel} ergeben. In der Schaltung braucht es dazu immer zwei unabhängige
Stellglieder.


\subsection{Spezielle Brückenschaltungen}

\subsubsection{Wheatstonesche Brücke}

\begin{figure}[H]
	\centering
	\begin{circuitikz}
		\usetikzlibrary{arrows.meta}

\tikzstyle{every node} = [font = \small]

\ctikzset{bipoles/thickness = 1}
\ctikzset{bipoles/length = 1cm}

\begin{scope}[line width = 1pt]
	\draw
		(3,-2.25) to[short, *-] ++(0,-0.75)
		\pgfextra{\ctikzset{bipoles/length = 1.4cm}}
		to[short] ++(-3.25,0) to[esource, name = U] ++(0,6)
		\pgfextra{\ctikzset{bipoles/length = 1cm}}
		to[short] ++(3.25,0) to[short, -*] ++(0,-0.75);
	\draw
		(3,2.25) to[short] ++(-1.5,0)
		to[R, l_=$R_x$] ++(0,-2.25)
		to[R, l_=$R_2$] ++(0,-2.25)
		to[short] ++(1.5,0);
	\draw
		(3,2.25) to[short] ++(1.5,0)
		to[pR, mirror, name=Pot, l=$\pfrac{\vphantom{\int}R_3}{\vphantom{\int}R_4}$] ++(0,-4.5)
		to[short] ++(-1.5,0);
	\draw
		(1.5,0) to[short, *-*] ++(1.25,0);
	\draw
		(3.25,0) to[short, *-] (Pot.wiper);
	\node
		at (-0.25,-0.05) {$\symup{U_{\! S}}$};
\end{scope}

	\end{circuitikz}
	\caption{Wheatstonesche Brückenschaltung.}
	\label{fig:wheatstone}
\end{figure}
Die Wheatstonesche Brücke enthält nur ohmsche Widerstände und kann daher \mbox{mit Gleich-} oder Wechselstrom
betrieben werden. Wird ein unbekannter Widerstand $R_x$ wie in Abbildung \ref{fig:wheatstone} eingebaut, ergibt
sich aus der Abgleichbedingung~\eqref{eqn:abgleich} folgender Ausdruck:
\begin{equation}
	R_x = R_2 \textstyle{\pfrac{\vphantom{\int}R_3}{\vphantom{\int}R_4}}
	\label{eqn:wheatstone}
\end{equation}
Da es zur Bestimmung von $R_x$ ausreicht, nur das Verhältnis von $R_3$ und $R_4$ zu variieren, können diese
beiden Bauteile durch ein Potentiometer ersetzt werden.

\subsubsection{Kapazitätsmessbrücke}

Ein idealer Kondensator ist vollständig durch seine Speicherkapazität beschrieben. In realen Kondensatoren
treten zusätzlich dielektrische Verluste auf, bei denen elektrische Energie in Wärme umgewandelt wird. Um
dies bei der Berechnung von Schaltungen zu berücksichtigen, wird im Ersatzschaltbild ein ohmscher Widerstand in
Reihe mit einem idealen Kondensator identischer Kapazität gesetzt. Der passende Operator lautet dann:
\begin{equation}
	Z_{C \, \text{real}} = R + \pfrac{1}{i \mkern1mu \omega C}
\end{equation}
\begin{figure}[H]
	\centering
	\begin{circuitikz}
		\usetikzlibrary{arrows.meta}

\tikzstyle{every node} = [font = \small]

\ctikzset{bipoles/thickness = 1}
\ctikzset{bipoles/length = 1cm}

\begin{scope}[line width = 1pt]
	\draw
		(3,-2.25) to[short, *-] ++(0,-0.75)
		\pgfextra{\ctikzset{bipoles/length = 1.4cm}}
		to[short] ++(-3.25,0) to[sV, l=$\symup{U_{\! S}}$] ++(0,6)
		\pgfextra{\ctikzset{bipoles/length = 1cm}}
		to[short] ++(3.25,0) to[short, -*] ++(0,-0.75);
	\draw
		(3,2.25) to[short] ++(-1.5,0)
		to[short] ++(0,-0.34)
		to[C, l_=$C_x$] ++(0,-0.41)
		to[R, l_=$R_x \,$] ++(0,-1.5)
		to[short] ++(0,-0.34)
		to[C, l_=$C_2$] ++(0,-0.41)
		to[R, name=Var, l_=$R_2 \,$] ++(0,-1.5)
		to[short] ++(1.5,0);
	\ctikztunablearrow{1}{1.25}{132.5}{Var}
	\draw
		(3,2.25) to[short] ++(1.5,0)
		to[pR, mirror, name=Pot, l=$\pfrac{\vphantom{\int}R_3}{\vphantom{\int}R_4}$] ++(0,-4.5)
		to[short] ++(-1.5,0);
	\draw
		(1.5,0) to[short, *-*] ++(1.25,0);
	\draw
		(3.25,0) to[short, *-] (Pot.wiper);
\end{scope}

	\end{circuitikz}
	\caption{Brückenschaltung zur Kapazitätsmessung.}
	\label{fig:kapazität}
\end{figure}
Eine Kapazitätsmessbrücke muss wie in Abbildung \ref{fig:kapazität} mit Wechselstrom gespeist werden. Um bei
der Messung von $C_x$ die Phasenverschiebung des Verlustwiderstandes $R_x$ zu kompensieren, wird zu $C_2$
noch ein weiterer verstellbarer Widerstand $R_2$ als zweiter Freiheitsgrad benötigt. Nach dem Abgleich
für Wechselstrombrücken~\eqref{eqn:wechsel} folgt:
\begin{align}
	C_x = C_2 \textstyle{\pfrac{\vphantom{\int}R_4}{\vphantom{\int}R_3}} &&
	R_x = R_2 \textstyle{\pfrac{\vphantom{\int}R_3}{\vphantom{\int}R_4}}
	\label{eqn:kapazität}
\end{align}

\subsubsection{Induktivitätsmessbrücke}

Auch eine reale Induktivität wandelt magnetische Feldenergie teilweise in Wärme um. Um also etwa eine Spule
darzustellen, wird wieder ein ohmscher Widerstand in Reihe mit einer idealen Induktivität geschaltet.
Ihr Widerstandsoperator kann wie folgt geschrieben werden:
\begin{equation}
	Z_{L \, \text{real}} = R + i \mkern1mu \omega L
\end{equation}

Wie zuvor kann nach dem Aufbau in Abbildung \ref{fig:induktivität} in die bekannte Abgleichbedingung für
Wechselstrom~\eqref{eqn:wechsel} eingesetzt werden. Lösen dieser Gleichung liefert:
\begin{align}
	L_x = L_2 \textstyle{\pfrac{\vphantom{\int}R_3}{\vphantom{\int}R_4}} &&
	R_x = R_2 \textstyle{\pfrac{\vphantom{\int}R_3}{\vphantom{\int}R_4}}
	\label{eqn:induktivität}
\end{align}
Um verlustbehaftete Bauteile auszumessen, dürfen entsprechende bekannte Komponenten $L_2$ oder $C_2$
nur möglichst geringe Fehler aufweisen. Für Induktivitäten ist dies besonders bei niedrigen Frequenzen $\nu$
kaum möglich. Statt einer in Reihe geschalteten Spule bietet es sich daher an, einen parallel geschalteten
Kondensator zu nutzen, da solche mit deutlich geringeren Verlusten gebaut werden können. Eine Schaltung dieser
Form heißt Maxwell-Brücke.

\begin{figure}
	\centering
	\begin{circuitikz}
		\usetikzlibrary{arrows.meta}

\tikzstyle{every node} = [font = \small]

\ctikzset{bipoles/thickness = 1}
\ctikzset{bipoles/length = 1cm}
\ctikzset{american inductors}

\begin{scope}[line width = 1pt]
	\draw
		(3,-2.25) to[short, *-] ++(0,-0.75)
		\pgfextra{\ctikzset{bipoles/length = 1.4cm}}
		to[short] ++(-3.25,0) to[sV, l=$\symup{U_{\! S}}$] ++(0,6)
		\pgfextra{\ctikzset{bipoles/length = 1cm}}
		to[short] ++(3.25,0) to[short, -*] ++(0,-0.75);
	\draw
		(3,2.25) to[short] ++(-1.5,0)
		to[short] ++(0,-0.2)
		to[L, mirror, l_=$L_x \,$] ++(0,-0.9)
		to[R, l_=$R_x \,$] ++(0,-1.15)
		to[short] ++(0,-0.2)
		to[L, mirror, l_=$L_2 \,$] ++(0,-0.9)
		to[R, name=Var, l_=$R_2 \,$] ++(0,-1.15)
		to[short] ++(1.5,0);
	\ctikztunablearrow{1}{1.25}{132.5}{Var}
	\draw
		(3,2.25) to[short] ++(1.5,0)
		to[pR, mirror, name=Pot, l=$\pfrac{\vphantom{\int}R_3}{\vphantom{\int}R_4}$] ++(0,-4.5)
		to[short] ++(-1.5,0);
	\draw
		(1.5,0) to[short, *-*] ++(1.25,0);
	\draw
		(3.25,0) to[short, *-] (Pot.wiper);
\end{scope}

	\end{circuitikz}
	\caption{Brückenschaltung zur Induktivitätsmessung.}
	\label{fig:induktivität}
\end{figure}

\subsubsection{Maxwell-Brücke}

\begin{figure}[H]
	\centering
	\begin{circuitikz}
		\usetikzlibrary{arrows.meta}

\tikzstyle{every node} = [font = \small]

\ctikzset{bipoles/thickness = 1}
\ctikzset{bipoles/length = 1cm}
\ctikzset{american inductors}

\begin{scope}[line width = 1pt]
	\draw
		(3,-2.25) to[short, *-] ++(0,-0.75)
		\pgfextra{\ctikzset{bipoles/length = 1.4cm}}
		to[short] ++(-3.25,0) to[sV, l=$\symup{U_{\! S}}$] ++(0,6)
		\pgfextra{\ctikzset{bipoles/length = 1cm}}
		to[short] ++(3.25,0) to[short, -*] ++(0,-0.75);
	\draw
		(3,2.25) to[short] ++(-1.5,0)
		to[short] ++(0,-0.25)
		to[L, mirror, l_=$L_x \,$] ++(0,-0.95)
		to[R, l_=$R_x \,$] ++(0,-1.3)
		to[R, l_=$R_2 \,$] ++(0,-2)
		to[short] ++(1.5,0);
	\draw
		(3,2.25) to[short] ++(1.5,0)
		to[R, name=Var3, l=$\, R_3$] ++(0,-2.5)
		to[C, -*, l_=$C_4$] ++(0,-2);
		\ctikztunablearrow{1}{1.25}{132.5}{Var3}
	\draw
		(4.5,-0.25) to[short] ++(1.5,0)
		to[R, name=Var4, l=$\, R_4$] ++(0,-2)
		to[short] ++(-3,0);
		\ctikztunablearrow{1}{1.25}{132.5}{Var4}
	\draw
		(1.5,-0.25) to[short, *-*] ++(1.25,0);
	\draw
		(3.25,-0.25) to[short, *-*] ++(1.25,0);
\end{scope}

	\end{circuitikz}
	\caption{Maxwell-Brückenschaltung zur Induktivitätsmessung.}
	\label{fig:maxwell}
\end{figure}
Hier müssen $R_3$ und $R_4$ nach Abbildung \ref{fig:maxwell} wieder separat anpassbar sein. Um
parallel geschaltete Widerstände und Kapazitäten zusammenzufassen, wird die Summe ihrer Kehrwerte
gebildet. Der komplexe Operator dazu lautet:
\begin{equation}
	Z_{RC} = \! \left( \pfrac{1}{R} + i \mkern1mu \omega C \right)^{\! -1}
\end{equation}
Zum Abgleich der Schaltung wird dies wieder in~\eqref{eqn:wechsel} eingesetzt und \mbox{anschließend eliminiert:}
\begin{align}
	L_x = R_2 R_3 C_4 && R_x = \textstyle{\pfrac{\vphantom{\int}R_2 R_3}{\vphantom{\int}R_4}}
	\label{eqn:maxwell}
\end{align}

\subsubsection{Wien-Robinson-Brücke}

\begin{figure}
	\centering
	\begin{circuitikz}
		\usetikzlibrary{arrows.meta}

\tikzstyle{every node} = [font = \small]

\ctikzset{bipoles/thickness = 1}
\ctikzset{bipoles/length = 1cm}
\ctikzset{american inductors}

\begin{scope}[line width = 1pt]
	\draw
		(3,-2.25) to[short, *-] ++(0,-0.75)
		\pgfextra{\ctikzset{bipoles/length = 1.4cm}}
		to[short] ++(-3.25,0) to[sV, l=$\symup{U_{\! S}}$] ++(0,6)
		\pgfextra{\ctikzset{bipoles/length = 1cm}}
		to[short] ++(3.25,0) to[short, -*] ++(0,-0.75);
	\draw
		(3,2.25) to[short] ++(-1.5,0)
		to[R, l_=\raisebox{-0.25ex}{$2R'$}] ++(0,-2.5)
		to[R, l_=\raisebox{-0.25ex}{$R'$}] ++(0,-2)
		to[short] ++(1.5,0);
	\draw
		(3,2.25) to[short] ++(1.5,0)
		to[short] ++(0,-0.4)
		to[C, l=\raisebox{-0.25ex}{$C$}] ++(0,-0.5)
		to[R, l=\raisebox{-0.25ex}{$R$}] ++(0,-1.6)
		to[C, -*, l_=\raisebox{-0.25ex}{$C$}] ++(0,-2);
	\draw
		(4.5,-0.25) to[short] ++(1.5,0)
		to[R, l=\raisebox{-0.25ex}{$R$}] ++(0,-2)
		to[short] ++(-3,0);
	\draw
		(1.5,-0.25) to[short, *-*] ++(1.25,0);
	\draw
		(3.25,-0.25) to[short, *-*] ++(1.25,0);
\end{scope}

	\end{circuitikz}
	\caption{Frequenzabhängige Wien-Robinson-Brückenschaltung.}
	\label{fig:wien}
\end{figure}

Wie in Abbildung \ref{fig:wien} dargestellt, enthält eine Wien-Robinson-Brücke keine verstellbaren
Widerstände mehr. Der Abgleich dieser Schaltung ist nur durch Einstellen genau einer bestimmten Frequenz
$\omega_0 = 2\pi \nu_0$ möglich. Durch Zusammenfassen der Bauteile und Einsetzen in den
Ausdruck~\eqref{eqn:bridge} lässt sich das Verhältnis von Brücken- und Speisespannung als
Betrag ausdrücken:
\begin{equation}
	\left| \symup{\pfrac{U_{\! Br}}{U_{\! S}}} \right|^2 \!\! =
	\pfrac{\left( \omega^2 R^2 C^2 - 1 \right)^{\! 2}}
	{9 \Bigl( \! \left( 1 - \omega^2 R^2 C^2 \right)^{\! 2} \! + 9 \omega^2 R^2 C^2 \Bigr)}
\end{equation}
Aus dieser Übertragungsfunktion lässt sich ablesen, dass $\symup{U_{\! Br}}$ für den Wert
$\omega_0 = (RC)^{-1}$ verschwindet. Bei niedrigeren und höheren Frequenzen ist die Schaltung durchlässig.
Die Wien-Robinson-Brücke entspricht daher einem elektronischen Sperrfilter. Zur weiteren Vereinfachung wird
noch folgendes Verhältnis eingeführt:
\begin{align}
	\Omega = \pfrac{\omega}{\raisebox{0.5ex}{$\omega_0$}} = \pfrac{\nu}{\raisebox{0.5ex}{$\nu_0$}} &&
	\left| \symup{\pfrac{U_{\! Br}}{U_{\! S}}} \right|^2 \!\! = \pfrac{\, 1 \,}{\raisebox{0.5ex}{$9$}} \,
	\pfrac{\left( \Omega^2 - 1 \right)^{\! 2}}
	{\left( 1 - \Omega^2 \right)^{\! 2} \! + 9 \Omega^2}
	\label{eqn:wien}
\end{align}
Ein realer Generator erzeugt keine reine Sinusschwingung. Das Verhältnis der Oberwellen zur Grundwelle
wird \textbf{Klirrfaktor} $k$ genannt und bemisst die Güte eines Sinusgenerators. Kleinere Faktoren zeigen
eine entsprechend höhere Qualität an. Mit der Wien-Robinson-Brücke kann durch Einstellen des Generators auf
die Sperrfrequenz $\nu_0$ \mbox{eine Messung} des Klirrfaktors durchgeführt werden. Die Amplitude der
Oberschwingung zur Frequenz $(n+1) \, \nu_0$ sei dabei als $U_{\! n} \!$ definiert. Dann gilt:
\begin{equation}
	k = \pfrac{\sqrt{\textstyle{\sum\nolimits_{n = 1}^N U_{\! n}^2}}}{U_{\! 0}}
	\label{eqn:klirr}
\end{equation}


\subsection{Fehlerrechnung}
\Umathoverbarrule\displaystyle=0.4pt
\Umathoverbarvgap\displaystyle=1.4pt
\Umathoverbarrule\textstyle=0.4pt
\Umathoverbarvgap\textstyle=1.4pt
Um die Abweichung der Messgrößen zu untersuchen, werden noch einige weitere Formeln benötigt.
Der Mittelwert $\overline{x}$ berechnet sich nach:
\begin{equation}
	\overline{x} = \pfrac{1}{N \,} \sum_{n=1}^N x_n
	\label{eqn:mittel}
\end{equation}
Zur Bestimmung der Standardabweichung $\symup{\Delta}\overline{x}$ wird folgender Ausdruck verwendet:
\begin{equation}
	(\symup{\Delta}\overline{x})^2 = \pfrac{1}{N(N-1)} \sum_{n=1}^N (x_n \! - \overline{x})^2
	\label{eqn:std}
\end{equation}
Durch die Gaußsche Fehlerfortpflanzung ist die Abweichung $\symup{\Delta}f$ für aus fehlerbehafteten
Messwerten $x_n \!$ abgeleitete Größen $f$ berücksichtigt:
\begin{equation}
	(\symup{\Delta}f)^2 = \sum_{n=1}^N
	\left( \! \pfrac{\partial^{\!} f}{\partial x_{\raisebox{0.2ex}{$\scriptstyle{n}$}}} \!
	\right)^{\!\! 2} \!\! (\symup{\Delta}x_{\raisebox{0.2ex}{$\scriptstyle{n}$}})^2
	\label{eqn:gauss}
\end{equation}
Für Größen der speziellen Formen $f = xyz$ oder $f = xyz^{-1}$ ergibt sich damit der Fehler:
\begin{equation}
	\symup{\Delta}f = f \, 
	\sqrt{\left( \! \pfrac{\symup{\Delta}x}{\raisebox{0.25ex}{$x$}} \! \right)^{\!\! 2} \!\! + \!
	\left( \! \pfrac{\symup{\Delta}y^{\,}}{\raisebox{1ex}{$y$}} \! \right)^{\!\! 2} \!\! + \!
	\left( \! \pfrac{\symup{\Delta}z}{\raisebox{0.25ex}{$z$}} \! \right)^{\!\! 2} \, }
	\label{eqn:speziell}
\end{equation}
