% Diskussion: Resultate mit Fehler/Genauigkeit zusammenstellen, Literaturwerte/Messmethoden/Ursachen vergleichen
% Literatur: Verwendete Literatur/Grafiken/Werte/Programme
% Anhang: Kopie der analog eingetragenen Messdaten

\vfill

\section{Diskussion}
\label{sec:diskussion}

Sowohl die Schaltung des Lock-In\hspace{0.1ex}-\hspace{-0.25ex}Verstärkers als auch das digitale Oszilloskop
weisen nur sehr geringe baubedingte Abweichungen auf. Die größte Fehlerquelle dürfte daher vom menschlichen
Faktor herrühren, also aus ungenauer Einstellung der Messapparatur oder dem fehlerhaften Ablesen der Werte.
Da die Messergebnisse insgesamt aber zu einem hohen Grad mit den Erwartungen aus der Theorie übereinstimmen,
kann für den Versuch von einer großen Gesamtgenauigkeit ausgegangen werden. 

\newpage

Das Verhalten des Gleichrichters lässt sich graphisch nur ohne Rauschen beurteilen.
In diesem Fall wird deutlich, dass der Spannungsverlauf aus sinusartigen Halbwellen genau mit
der Vorhersage im Einklang steht. Auch die phasenabhängige Intensität der Ausgangsspannung passt bis auf leichte
Verschiebungen zur Theorie. Die fest auftretende Verschiebung $d$ ist am besten durch ungenaues Einstellen
des Spannungsmessers erklärt. Der Versatz $c$ kann auf ungewollte innere Widerstände
des Aufbaus zurückgeführt werden, die eine geringe Phasendifferenz hervorrufen. Zu beachten ist bei der
Analyse des störbehafteten Intensitätsverlaufs noch das Auftreten signifikanter Schwankungen in der
Gleichspannung. Trotzdem stimmt das Ergebnis gut mit dem eines reinen Sinussignals überein. Daraus
lässt sich auf die Güte des Lock-In\hspace{0.1ex}-\hspace{-0.25ex}Verstärkers schließen. Dieser ist offenbar
dazu in der Lage, aus einem stark verrauschten Signal mit hoher Genauigkeit den Anteil der Modulationsfrequenz
herauszufiltern und zu verstärken.

Auch bei der Anwendung zur Rauschunterdrückung in der Photodetektorschaltung wird der Nutzen dieser
Funktion deutlich. Trotz des ambienten Lichteinfalls passen die Messdaten sehr gut zum
korrigierten Ausgleich mit der Theoriefunktion. Die Abweichung für nähere Abstände rührt hauptsächlich
daher, dass Höhe und Neigung des Detektors nicht perfekt mit der Leuchtdiode abgestimmt sind. 

\vfill

Alle Informationen sind aus der Versuchsanleitung \cite{lock-in} entnommen. Auch die Skizzen
der exemplarischen Spannungsverläufe sowie die schematischen Schaltbilder sind an die darin enthaltenen
Grafiken angelehnt.

\vfill
\vfill
\vfill
\vfill
\vfill
\vfill
\vfill
