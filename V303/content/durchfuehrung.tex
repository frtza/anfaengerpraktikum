% Aufgabe: Messaufgaben auflisten
% Vorbereitung: Vorbereitungsaufgaben bearbeiten
% Versuchsaufbau: Verwendete Apparatur, Beschreibung Funktionsweise/Nutzen mit Skizze/Foto
\section{Durchführung}
\label{sec:durchführung}

Am Lock-In\hspace{0.1ex}-\hspace{-0.25ex}Verstärker werden zuerst die Signale der Ausgänge des
Funktionsgenerators (Reference/Oscillator) gemessen. An \scriptsize\textbf{OUT}\normalsize{} lässt sich eine
Spannung $U_{\! \text{sig}}$ mit variabler Amplitude als Nutzsignal abgreifen. Das Referenzsignal
$U_{\! \text{ref}}$ wird mit einer konstanten Amplitude \qty{5}{\volt} durch
\textbf{\small{Phase \scriptsize{OUT}}}\normalsize{} geliefert. Beide Spannungen entsprechen Sinussignalen mit
identischer Kreisfrequenz $\omega = 2\pi\nu$. Der Vorverstärker sowie der Detektor
werden auf eine Verstärkung (Gain) von \num{10} kalibriert und gemeinsam mit dem eingebauten
digitalen Oszilloskop auf Gleichspannung eingestellt. Dadurch lässt
sich das unerwartete Filtern bestimmter Frequenzen vermeiden.

\subsection{Verifikation der Phasenabhängigkeit}

\begin{figure}[H]
	\centering
	\begin{tikzpicture}
		\usetikzlibrary{arrows.meta, calc}

\tikzstyle{every node} = [font = \footnotesize\bfseries]

\definecolor{steelblue}{HTML}{4682B4}
\definecolor{salmon}{HTML}{FA8072}
\definecolor{olivedrab}{HTML}{6B8E23}
\definecolor{golden}{HTML}{EDC824}

\coordinate (pos) at (1,1);
\draw
	(pos) -- ++(0,1.8) -- ++(2,0) -- ++(0,-1.8) -- cycle;
\node[align=center] at ($(pos) + (1,1.35)$)
	{\textcolor{steelblue}{\begin{tabular}{c} N$\hspace{-0.1ex}$oise \\[-0.5ex] Generator \end{tabular}}};
\coordinate (b1) at ($(pos) + (0.55,0.35)$);
\coordinate (b2) at ($(pos) + (1.45,0.35)$);

\coordinate (pos) at ($(pos) + (3,0)$);
\draw
	(pos) -- ++(0,1.8) -- ++(2,0) -- ++(0,-1.8) -- cycle;
\node[align=center] at ($(pos) + (1,1.5)$)
	{\textcolor{steelblue}{Ref.}\hspace{1.5ex}\textcolor{salmon}{Osc.}};
\coordinate (a1) at ($(pos) + (0.55,0.35)$);
\coordinate (a2) at ($(pos) + (1.45,0.35)$);

\coordinate (pos) at ($(pos) + (3.5,0)$);
\draw
	(pos) -- ++(0,1.8) -- ++(2,0) -- ++(0,-1.8) -- cycle;
\node[align=center] at ($(pos) + (1,1.35)$)
	{\textcolor{salmon}
	{\begin{tabular}{c} P$\hspace{-0.15ex}$hase \\[-0.5ex] 
	S$\hspace{-0.1ex}$hif$\hspace{0.15ex}$ter \end{tabular}}};
\coordinate (e1) at ($(pos) + (0.55,0.35)$);
\coordinate (e2) at ($(pos) + (1.45,0.35)$);

\coordinate (pos) at ($(pos) + (4.2,0)$);
\draw
	(pos) -- ++(0,2.4) -- ++(2.8,0) -- ++(0,-2.4) -- cycle;
\node[align=center] at ($(pos) + (1.4,2.1)$) {Oscilloscope};
\coordinate (h1) at ($(pos) + (0.95,0.35)$);
\coordinate (h2) at ($(pos) + (1.85,0.35)$);

\coordinate (pos) at ($(pos) + (0.4,0.8)$);
\draw[golden, line width=0.5pt] ($(pos) + (0,0.6)$)
	sin ++(0.2,0.2) cos ++(0.2,-0.2)
	sin ++(0.2,-0.2) cos ++(0.2,0.2)
	sin ++(0.2,0.2) cos ++(0.2,-0.2)
	sin ++(0.2,-0.2) cos ++(0.2,0.2)
	sin ++(0.2,0.2) cos ++(0.2,-0.2);
\draw[olivedrab, line width=0.5pt] ($(pos) + (0,0.2)$) -- ++(2,0);
\draw
	(pos) -- ++(0,1) -- ++(2,0) -- ++(0,-1) -- cycle;

\coordinate (pos) at ($(1,1) + (1.6,3.2)$);
\draw
	(pos) -- ++(0,1.8) -- ++(1.8,-0.9) -- cycle;
\node[align=center] at ($(pos) + (-1,1.65)$)
	{\textcolor{steelblue}{Pre\hspace{0.3ex}-Amp}};
\coordinate (c1) at ($(pos) + (0.25,0.55)$);
\coordinate (c2) at ($(pos) + (0.25,1.25)$);
\coordinate (c3) at ($(pos) + (1.25,0.9)$);

\coordinate (pos) at ($(pos) + (3.15,0)$);
\draw
	(pos) -- ++(0,1.8) -- ++(2,0) -- ++(0,-1.8) -- cycle;
\node[align=center] at ($(pos) + (1,1.5)$)
	{\textcolor{steelblue}{F$\hspace{-0.1ex}$ilter}};
\coordinate (d1) at ($(pos) + (0.55,0.35)$);
\coordinate (d2) at ($(pos) + (1.45,0.35)$);

\coordinate (pos) at ($(pos) + (3,0)$);
\draw
	(pos) -- ++(0,1.8) -- ++(2.75,0) -- ++(0,-1.8) -- cycle;
\node[align=center] at ($(pos) + (1.375,1.5)$)
	{Detector};
\coordinate (f1) at ($(pos) + (0.55,0.35)$);
\coordinate (f2) at ($(pos) + (1.375,0.35)$);
\coordinate (f3) at ($(pos) + (2.2,0.35)$);

\coordinate (pos) at ($(pos) + (3.75,0)$);
\draw
	(pos) -- ++(0,1.8) -- ++(2,0) -- ++(0,-1.8) -- cycle;
\node[align=center] at ($(pos) + (1,1.35)$)
	{\begin{tabular}{c} Low\hspace{0.1ex}-P$\hspace{-0.2ex}$ass
	\\[-0.5ex] A$\hspace{-0.1ex}$mplif$\hspace{0.2ex}$ier \end{tabular}};
\coordinate (g1) at ($(pos) + (0.55,0.35)$);
\coordinate (g2) at ($(pos) + (1.45,0.35)$);

\draw[steelblue, line width=0.6pt]
	(a1) -- ($(a1) + (0,-0.7)$) -- ($(b2) + (0,-0.7)$) -- (b2);
\draw[steelblue, line width=0.6pt]
	(b1) -- ($(b1) + (0,-0.7)$) -- ++(-0.9,0) -- ($(c2) + (-2.2,0)$) -- (c2);
\draw[steelblue, line width=0.6pt]
	(c3) -- ($(c3) + (0,-1.25)$) -- ($(d1) + (0,-0.7)$) -- (d1);
\draw[steelblue, line width=0.6pt]
	(d2) -- ($(d2) + (0,-0.7)$) -- ($(f1) + (0,-0.7)$) -- (f1);

\draw[salmon, line width=0.6pt]
	(a2) -- ($(a2) + (0,-0.7)$) -- ($(e1) + (0,-0.7)$) -- (e1);
\draw[salmon, line width=0.6pt]
	(e2) -- ($(e2) + (0,-0.7)$) -- ($(f2) + (0,-3.9)$) -- (f2);

\draw[golden, line width=0.6pt]
	(f3) -- ($(f3) + (0,-3.9)$) -- ($(h1) + (0,-0.7)$) -- (h1);
\draw[golden, line width=0.6pt]
	($(f3) + (0,-0.7)$) -- ($(g1) + (0,-0.7)$) -- (g1);

\draw[olivedrab, line width=0.6pt]
	(g2) -- ($(g2) + (0,-0.7)$) -- ++(1,0) -- ++(0,-3.2) -- ($(h2) + (0,-0.7)$) -- (h2);

\node[circle, fill=black, inner sep=0pt, minimum size=0.75ex] at (a1) {};
\node[circle, fill=black, inner sep=0pt, minimum size=0.75ex] at (a2) {};
\node[circle, fill=black, inner sep=0pt, minimum size=0.75ex] at (b1) {};
\node[circle, fill=black, inner sep=0pt, minimum size=0.75ex] at (b2) {};
\node[circle, fill=black, inner sep=0pt, minimum size=0.75ex] at (c1) {};
\node[circle, fill=black, inner sep=0pt, minimum size=0.75ex] at (c2) {};
\node[circle, fill=black, inner sep=0pt, minimum size=0.75ex] at (c3) {};
\node[circle, fill=black, inner sep=0pt, minimum size=0.75ex] at (d1) {};
\node[circle, fill=black, inner sep=0pt, minimum size=0.75ex] at (d2) {};
\node[circle, fill=black, inner sep=0pt, minimum size=0.75ex] at (e1) {};
\node[circle, fill=black, inner sep=0pt, minimum size=0.75ex] at (e2) {};
\node[circle, fill=black, inner sep=0pt, minimum size=0.75ex] at (f1) {};
\node[circle, fill=black, inner sep=0pt, minimum size=0.75ex] at (f2) {};
\node[circle, fill=black, inner sep=0pt, minimum size=0.75ex] at (f3) {};
\node[circle, fill=black, inner sep=0pt, minimum size=0.75ex] at ($(f3) + (0,-0.7)$) {};
\node[circle, fill=black, inner sep=0pt, minimum size=0.75ex] at (g1) {};
\node[circle, fill=black, inner sep=0pt, minimum size=0.75ex] at (g2) {};
\node[circle, fill=black, inner sep=0pt, minimum size=0.75ex] at (h1) {};
\node[circle, fill=black, inner sep=0pt, minimum size=0.75ex] at (h2) {};

\node at ($(a1) + (0,0.2)$) {\scalebox{0.55}{$\text{OUT}$}};
\node at ($(a2) + (0,0.2)$) {\scalebox{0.55}{$\text{OUT}$}};
\node at ($(b1) + (0,0.2)$) {\scalebox{0.55}{$\text{OUT}$}};
\node at ($(b2) + (0,0.2)$) {\scalebox{0.55}{$\text{IN}$}};
\node at ($(d1) + (0,0.2)$) {\scalebox{0.55}{$\text{IN}$}};
\node at ($(d2) + (0,0.2)$) {\scalebox{0.55}{$\text{OUT}$}};
\node at ($(e1) + (0,0.2)$) {\scalebox{0.55}{$\text{IN}$}};
\node at ($(e2) + (0,0.2)$) {\scalebox{0.55}{$\text{OUT}$}};
\node at ($(f1) + (0,0.2)$) {\scalebox{0.55}{$\text{IN}$}};
\node at ($(f2) + (0,0.2)$) {\scalebox{0.55}{$\text{IN}$}};
\node at ($(f3) + (0,0.2)$) {\scalebox{0.55}{$\text{OUT}$}};
\node at ($(g1) + (0,0.2)$) {\scalebox{0.55}{$\text{IN}$}};
\node at ($(g2) + (0,0.2)$) {\scalebox{0.55}{$\text{OUT}$}};
\node at ($(h1) + (0,0.2)$) {\scalebox{0.55}{$\text{CH$\hspace{0.3ex}$1}$}};
\node at ($(h2) + (0,0.2)$) {\scalebox{0.55}{$\text{CH$\hspace{0.3ex}$2}$}};

\node at ($(a2) + (0,0.4)$) {\scalebox{0.55}{$\text{P$\hspace{-0.15ex}$hase}$}};
\node at ($(b2) + (0,0.4)$) {\scalebox{0.55}{$\text{S$\hspace{-0.05ex}$ignal}$}};
\node at ($(c1) + (0.275,-0.04)$) {$\symbf{-}$};
\node at ($(c2) + (0.275,0)$) {$\symbf{+}$};
\node at ($(d2) + (0,0.4)$) {\scalebox{0.55}{$\text{Band$\hspace{0.1ex}$-Pass}$}};
\node at ($(f1) + (0,0.4)$) {\scalebox{0.55}{$\text{S$\hspace{-0.05ex}$ignal}$}};
\node at ($(f2) + (0,0.417)$) {\scalebox{0.55}{$\text{Reference}$}};

\draw[line width=0.45pt]
	(c1) -- ++(-1,0) -- ++(0,-0.75);
\draw[line width=0.45pt]
	($(c1) + (-1.25,-0.75)$) -- ++(0.5,0);
\draw[line width=0.45pt]
	($(c1) + (-1.175,-0.8125)$) -- ++(0.35,0);
\draw[line width=0.45pt]
	($(c1) + (-1.1,-0.875)$) -- ++(0.2,0);

	\end{tikzpicture}
	\vspace{1.23ex}
	\caption{Schematischer Aufbau eines Lock-In\hspace{0.1ex}-\hspace{-0.25ex}Verstärkers.}
	\label{fig:lock-in}
\end{figure}

\enlargethispage{\baselineskip}

Abbildung \ref{fig:lock-in} zeigt das Schaltnetz eines Lock-In\hspace{0.1ex}-\hspace{-0.25ex}Verstärkers
zusammen mit Signalwegen, die der Farbkodierung in Abbildung \ref{fig:verlauf} folgen. Angesetzt wird eine
Frequenz von $\nu = \qty{1}{\kilo\hertz}$. Das Nutzsignal $U_{\! \text{sig}}$ hat eine Amplitude von
\qty{10}{\milli\volt} und ist durch das anfangs phasengleiche Sinussignal $U_{\! \text{ref}}$ moduliert. Der
Rauschgenerator bleibt zunächst überbrückt. Um die Funktion des im Detektor enthaltenen phasenempfindlichen
Gleichrichters zu beurteilen, werden die zeitlichen Verläufe des Mischsignals $U_{\! \text{mix}}$ am digitalen
Oszilloskop unter Variation der relativen Phase als Bildschirmaufnahme mitgeführt. Dabei ist die Anzeige des
Spannungsmessers vertikal auf \qty{200}{\milli\volt}$\!$/div und horizontal auf \qty{200}{\micro\second}/div
skaliert. Weiter wird die integrierte Gleichspannung $U_{\! \text{out}}$ für verschiedene Phasenverschiebungen
nachgehalten. Um eine möglichst geringe Bandbreite der verbleibenden Oberfrequenzen zu garantieren, ist am
Tiefpass eine vergleichsweise hohe Integrationszeit von $\tau = \qty{3}{\second}$ eingestellt. Im Anschluss
an diese Messungen nutzt man den Störgenerator, um $U_{\! \text{sig}}$ ein Rauschsignal der Größenordnung
\qty{10}{\milli\volt} zuzufügen. Das vorherige Vorgehen wird für diesen Fall wiederholt, um im Vergleich die
Güte der Rauschunterdrückung des Lock-In\hspace{0.1ex}-\hspace{-0.25ex}Verstärkers nachzuweisen.

\subsection{Anwendung zur Rauschunterdrückung}

\begin{figure}[H]
	\centering
	\begin{tikzpicture}
		\input{content/sketch_b.tex}
	\end{tikzpicture}
	\vspace{1.23ex}
	\caption{Schematischer Aufbau einer Photodetektorschaltung.}
	\label{fig:photo}
\end{figure}

Wie in Abbildung \ref{fig:photo} gezeigt, ist nun eine Leuchtdiode (LED) verbaut. Sie wird bei
\qty{10}{\volt} mit einer Frequenz von $\nu = \qty{100}{\hertz}$ moduliert. Ein Photodetektor (PD) registriert
ihr Lichtsignal. Beide Bauteile sind auf einer Schiene geführt. Durch gegenseitiges Verschieben kann
der Abstand $r$ zwischen LED und PD variiert werden. Um die abstandsabhängige Intensitätsänderung
zu messen, muss die bekannte Signalfrequenz der Lichtquelle vom Hintergrundrauschen der Umgebungsbeleuchtung
getrennt werden. Dazu kommt der Lock-In\hspace{0.1ex}-\hspace{-0.25ex}Verstärker zum Einsatz, indem die Werte
der Ausgangsspannung $U_{\! \text{out}}$ zu verschiedenen Entfernungen aufgenommen und verarbeitet werden.
