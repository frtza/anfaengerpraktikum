% Titel: Ziel des Versuchs
% Theorie: Physikalische Grundlagen von Versuch/Messverfahren, Gleichungen ohne Herleitung knapp erklären
\section{Theorie}
\label{sec:theorie}

Lock-In\hspace{0.1ex}-\hspace{-0.25ex}Verstärker finden Verwendung in der Messung stark
verrauschter Signale, indem diese mit einer Referenzfrequenz $\omega_0$ moduliert
werden. Aus dem Messsignal entsteht $U_{\! \text{sig}}$ als Nutzsignal, welches unter
Anwendung eines Bandpassfilters von Rauschanteilen mit niedrigen ($\omega\ll\omega_0$)
und hohen ($\omega\gg\omega_0$) Frequenzen bereinigt wird. Im nachgeschalteten Mischer
wird $U_{\! \text{sig}}$ mit einer Referenzspannung $U_{\! \text{ref}}$ der Frequenz
$\omega_0$ multipliziert, wobei durch Variation deren Phase eine Synchronisation
($\phi = 0$) beider Signale erreicht wird. Zuletzt wird das Mischsignal
$U_{\! \text{mix}} \! = U_{\! \text{sig}} \! \times U_{\! \text{ref}}$ über
einen Tiefpass integriert. Durch Wahl einer ausreichend großen Zeitkonstante
($\tau \gg 1/\omega_0$) kann diese Integration über mehrere Perioden der Modulationsfrequenz
stattfinden. So werden asynchrone Rauschbeiträge weitestgehend herausgemittelt. Es ergibt sich
mit $U_{\! \text{out}}$ eine Gleichspannung, deren Wert proportional zum Eingangssignal ist.
Unter der Farbkodierung nach Abbildung \ref{fig:verlauf} lässt sich in Abbildung \ref{fig:lock-in}
der schematische Aufbau eines Lock-In\hspace{0.1ex}-\hspace{-0.25ex}Verstärkers nachvollziehen.
\par
Die Bandbreite $\symup{\Delta}\nu$ des Restrauschens ist vom Integrierglied abhängig. Durch Wahl
einer besonders großen Zeitkonstante $\tau = RC$ am Tiefpass, lässt sich
$\symup{\Delta}\nu = 1/(\pi RC)$ beliebig klein kalibrieren.
Dieses Vorgehen erlaubt das Erreichen von Gütefaktoren bis $Q = \num{100000}$. Ein einfacher Bandpass
besitzt dagegen eine Güte von Q = \num{1000}.

\begin{figure}[H]
	\centering
	\includegraphics{build/graphic.pdf}
	\caption{Exemplarische Signalverläufe am Lock-In\hspace{0.1ex}-\hspace{-0.25ex}Verstärker.}
	\label{fig:verlauf}
\end{figure}

Anhand Abbildung \ref{fig:verlauf} lassen sich die mathematischen Hintergründe des Filtervorgangs
darstellen. Eine typische Signalspannung der Form
\begin{equation}
	U_{\! \text{sig}} \hspace{-0.1ex} = U_0 \sin(\omega t)
	\label{eqn:sig}
\end{equation}
wird hier durch ein Rechtecksignal gleicher Frequenz moduliert. Zu beachten ist dabei, dass die Amplitude dieser
Referenzspannung $U_{\! \text{ref}}$ normiert ist. Als Näherung wird ihre Fourierreihe aus den ungeraden
Harmonischen zur Grundfrequenz $\omega$ genutzt:
\begin{align}
	U_{\! \text{ref}} \hspace{0.1ex} &=
	\pfrac{4}{\raisebox{0.8ex}{$\pi$}} \!
	\left( \sin(\omega t) +
	\pfrac{1}{\raisebox{0.8ex}{$3$}} \sin(3\omega t) +
	\pfrac{1}{\raisebox{0.8ex}{$5$}} \sin(5\omega t) +
	\pfrac{1}{\raisebox{0.8ex}{$7$}} \sin(7\omega t) + \ldots \right) \label{eqn:ref} \\
	\intertext{Am Mischer folgt damit für das Produkt aus Signal und Referenz:}
	U_{\! \text{mix}} &=
	\pfrac{2}{\raisebox{0.8ex}{$\pi$}} \, U_0 \!
	\left( 1 -
	\pfrac{2}{\raisebox{0.8ex}{$3$}} \cos(2\omega t) -
	\pfrac{2}{\raisebox{0.8ex}{$15$}} \cos(4\omega t) -
	\pfrac{2}{\raisebox{0.8ex}{$35$}} \cos(6\omega t) - \ldots \right) \label{eqn:mix}
\end{align}
Nach diesem Schritt sind nur noch die geraden Oberwellen von $\omega$ enthalten, der Detektor funktioniert
demnach als Gleichrichter. Der Tiefpassfilter wird anschließend so gewählt, dass alle höheren Schwingungen
unterdrückt werden. Nun kann die Gleichspannung
\begin{equation}
	U_{\! \text{out}} \hspace{-0.1ex} = \pfrac{2}{\raisebox{0.8ex}{$\pi$}} \, U_0
\end{equation}
abgegriffen werden. Weisen $U_{\! \text{sig}}$ und $U_{\! \text{ref}}$ anders als in Abbildung
\ref{fig:verlauf} eine \mbox{Phasendifferenz $\phi$} auf, so ergibt sich für die Ausgangsspannung stattdessen:
\begin{equation}
	U_{\! \text{out}} \hspace{-0.1ex} = \pfrac{2}{\raisebox{0.8ex}{$\pi$}} \, U_0 \cos(\phi)
\end{equation}
Entspricht die Referenzspannung einem nicht normierten Sinussignal,
gilt eine allgemeine Proportionalität der folgenden Form:
\begin{equation}
	U_{\! \text{out}} \hspace{-0.1ex} \propto U_0 \cos(\phi)
	\label{eqn:out}
\end{equation}

\vfill

\section{Aufgabe}

Ziel dieser Versuchsreihe ist es, die Funktion eines Lock-In\hspace{0.1ex}-\hspace{-0.25ex}Verstärkers
zu untersuchen. Dazu werden zunächst die Phasenabhängkeiten der Gestalt des Mischsignals sowie des
\mbox{Intensitätsverlaufs} der Ausgangsspannung betrachtet. Anschließend wird die aufgebaute Messapparatur
zur Rauschunterdrückung in einer Photodetektorschaltung angewendet.

\newpage
