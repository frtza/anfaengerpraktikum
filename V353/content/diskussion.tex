% Diskussion: Resultate mit Fehler/Genauigkeit zusammenstellen, Literaturwerte/Messmethoden/Ursachen vergleichen
% Literatur: Verwendete Literatur/Grafiken/Werte/Programme
% Anhang: Kopie der analog eingetragenen Messdaten

\vfill

\section{Diskussion}
\label{sec:diskussion}

Insgesamt liegen die resultierenden Werte der verschiedenen Verfahren zur Bestimmung der Zeitkonstante
zwar innerhalb einer Größenordnung, weisen aber dennoch signifikante Abweichungen auf. So liefert die
Entladungskurve $RC = \input{build/t-U_RC.tex}$ als das kleinste Ergebnis mit dem geringsten Relativfehler
zum Modell. Dies ist darin begründet, dass anhand des Oszillographenschirms sehr genau Punkte
abgelesen werden können. Die große Differenz zu den anderen Methoden kann durch einen unbekannten Fehler
in der Feinskalierung erklärt werden. So liefert eine plausible Abweichung der Skala von nur
\qty{0.125}{\milli\second} bereits die Grenzen
$\input{build/t-U_RC_lower.tex} \leq RC \leq \input{build/t-U_RC_upper.tex}$.
Dagegen erscheint der Amplitudenverlauf robuster, da die Zeitkonstante
$RC = \input{build/f-U_RC_corr.tex}$ hier nicht durch eine potenziell ungenaue Skalierung der Apparatur
beeinflusst wird. Als einzige Annahme ist die Näherung $U_0 = \input{build/f-U_0.tex}$ als Messwert bei der
niedrigsten Frequenz $\nu = \qty{10}{\hertz}$ gegeben. Die Eingangsamplitude am Funktionsgenerator bleibt
ebenfalls über ein breites Frequenzspektrum stabil. Zuletzt wird durch die
Abweichung von $RC = \input{build/f-phi_RC_corr.tex}$ aus dem Phasenverlauf klar, dass die Messungen
stark streuen. Tatsächlich kann für diese Daten insgesamt von einer schlechten Qualität ausgegangen werden, da
das Verhalten zum Triggern des Oszilloskops nicht konsistent gesetzt wird und es somit zu Sprüngen kommt.
Diese Fehler werden auch im Polarplot deutlich. Abschließend dient die Betrachtung der integrierten Signale
noch als Bestätigung der hierzu postulierten Beziehung.

\newpage

$ $

\vfill

Alle theoretischen Grundlagen sind den Versuchsanleitungen \cite{brücke, relax} entnommen. Auch die Schaltkizze
sowie der schematische Spannungsverlauf sind an darin enthaltene Grafiken angelehnt.
