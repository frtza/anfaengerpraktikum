% Titel: Ziel des Versuchs
% Theorie: Physikalische Grundlagen von Versuch/Messverfahren, Gleichungen ohne Herleitung knapp erklären
\section{Theorie}
\label{sec:theorie}

\subsection{Relaxationsvorgänge}

Allgemein bezeichnen Relaxationserscheinungen das nichtoszillatorische Einstellen des
Gleichgewichtszustandes eines physikalischen Systems. Unter der Annahme, dass sich die
Änderungsrate dabei proportional zur Abweichung von $A(t)$ zum asymptotischen stationären
Endzustand $A_{\raisebox{0.25ex}{\(\scriptscriptstyle{\infty}\)}\!\!}$ verhält, lässt sich die Gleichung
\begin{equation}
	\pfrac{\symup dA}{\symup dt} = c \hspace{0.3ex} (A(t) -
	A_{\raisebox{0.25ex}{$\scriptscriptstyle{\infty}$}\!}) \label{eqn:rate_allgemein}
\end{equation}
mit der Konstante $c$ aufstellen. Integration von \eqref{eqn:rate_allgemein} über die
Zeitpunkte $0$ bis $t$ liefert
\begin{equation}
	\int\limits_{A_0}^{\raisebox{-0.25ex}{$\scriptstyle{A(t)}$}} \!\!
	\pfrac{\symup dA}{\hspace{0.15ex}A - A_{\raisebox{0.25ex}{$\scriptscriptstyle{\infty}$}}} = \!\!
	\int\limits_0^t \! c \, \symup dt' \stepcounter{equation}\tag{2a}
\end{equation}
als entsprechende Operationsvorschrift sowie die äquivalente Formulierung
\begin{equation}
	\ln \pfrac{A(t) - A_{\raisebox{0.25ex}{$\scriptscriptstyle{\infty}$}}}{A_0 -
	A_{\raisebox{0.25ex}{$\scriptscriptstyle{\infty}$}}} = c \hspace{0.15ex} t \tag{2b}
\end{equation}
für deren Ergebnis. Durch Auflösen nach der Größe $A(t)$ ergibt sich damit der Ausdruck
\begin{equation}
	A(t) = A_{\raisebox{0.25ex}{$\scriptscriptstyle{\infty}$}\!} + \hspace{0.15ex}
	(A_0 - A_{\raisebox{0.25ex}{$\scriptscriptstyle{\infty}$}\!}) \hspace{0.15ex}
	\exp (c \hspace{0.15ex} t) \tag{2c} \label{eqn:relax_allgemein}
\end{equation}
zur Beschreibung des Systems. Aus \eqref{eqn:relax_allgemein} folgt noch $c < 0$, sodass $A(t)$
beschränkt ist.

\subsection{Kondensatorladung}

Mit dem Ladungsträgerstrom $I$ und der lokalen Ladung $Q$ gilt die Beziehung
\begin{equation}
	\symup dQ = - I \symup dt
	\label{eqn:ladung}
\end{equation}
an jedem Punkt in der Schaltung. Nach dem ohmschen Gesetz kann außerdem
\begin{equation}
	U \hspace{-0.025ex} = RI
	\label{eqn:ohm}
\end{equation}
geschrieben werden, wobei die Spannung $U$ am Widerstand $R$ abfällt. Daraus folgt dann
\begin{equation}
	U \hspace{-0.025ex} = \pfrac{Q}{C} = RI = -R \,\pfrac{\symup dQ}{\symup dt}
	\stepcounter{equation}\tag{5a}
\end{equation}
als Differentialgleichung am Kondensator der Kapazität $C$. Umgeschrieben in die Form
\begin{equation}
	\pfrac{\symup dQ}{\symup dt} = -\pfrac{1}{RC} \, Q
	\tag{5b}
\end{equation}
entspricht dies nach \eqref{eqn:relax_allgemein} dem Zusammenhang
\begin{equation}
	Q(t) = Q_{\raisebox{0.25ex}{$\scriptscriptstyle{\infty}$}\!} + \hspace{0.15ex}
	(Q_0 - Q_{\raisebox{0.25ex}{$\scriptscriptstyle{\infty}$}\!}) \hspace{0.15ex}
	\exp \! \left(-\pfrac{1}{RC} \, \hspace{0.15ex} t\right)
	\label{eqn:relax_kondensator}
\end{equation}
für den zeitlichen Verlauf der Kondensatorladung.

\subsubsection{Entladevorgang}

Bei der Entladung fällt die anfängliche Ladung $Q_0$ gegen
$Q_{\raisebox{0.25ex}{$\scriptscriptstyle{\infty}$}\!} = 0$ ab. Laut \eqref{eqn:relax_kondensator}
ist dann
\begin{align}
	Q(t) &= Q_0 \exp \! \left(-\pfrac{1}{RC} \, \hspace{0.15ex} t\right)
	\stepcounter{equation}\tag{7a}
	\label{eqn:entladung_ladung} \\
	\intertext{gegeben. Gleichzeitig lässt sich mit $Q = CU$ und $Q_0 = CU_{\hspace{-0.15ex}0}$ der Ausdruck}
	U\hspace{-0.025ex}(t) &= U_{\!0} \exp \! \left(-\pfrac{1}{RC} \, \hspace{0.15ex} t\right)
	\tag{7b}
	\label{eqn:entladung_spannung}
\end{align}
für die Kondensatorspannung aufstellen.

\subsubsection{Aufladevorgang}

Anders als zuvor ist der Kondensator mit $Q_0 = 0$ zunächst entladen. Unter der angelegten
Gleichspannung $U_{\hspace{-0.15ex}0}$ gilt
$Q_{\raisebox{0.25ex}{$\scriptscriptstyle{\infty}$}\!} = CU_{\hspace{-0.15ex}0}$ als aufgeladener Endzustand.
Einsetzen in \eqref{eqn:relax_kondensator} liefert
\begin{align}
	Q(t) &= CU_{\hspace{-0.15ex}0} 
	\!\left(\!1 - \exp \! \left(-\pfrac{1}{RC} \, \hspace{0.15ex} t\right) \!\right)
	\stepcounter{equation}\tag{8a}
	\label{eqn:aufladung_ladung} \\
	\intertext{zur Beschreibung der Ladung. Weiter definiert der Term}
	U\hspace{-0.025ex}(t) &= U_{\!0}
	\!\left(\!1 - \exp \! \left(-\pfrac{1}{RC} \, \hspace{0.15ex} t\right) \!\right)
	\tag{8b}
	\label{eqn:aufladung_spannung}
\end{align}
den entsprechenden Spannungsverlauf.

\subsubsection{Zeitkonstante}

Mit $\tau = RC$ ist die charakteristische Zeitkonstante des RC\hspace{0.15ex}-Kreises bezeichnet.
Diese bemisst die Geschwindigkeit, mit welcher das System gegen den Endzustand strebt.

\newpage

\subsection{Periodische Auslenkung}

Wird nun stattdessen eine periodische Generatorspannung der Gestalt
$U_{\hspace{-0.025ex}\text{in}\!} = U_{\hspace{-0.025ex}0} \cos(\omega t)$ mit einer Kreisfrequenz
$\omega = 2\pi \hspace{0.15ex} \nu$ angelegt, bietet sich die Durchführung einer komplexen
Wechselstromrechnung an.

\subsubsection{Impedanzen}

Mit der imaginären Einheit $i \mkern1mu$ kann die Gesamtimpedanz der Schaltung als
\begin{equation}
	Z_{\hspace{-0.15ex}RC} = Z_{\hspace{-0.15ex}R} + Z_C = R + \pfrac{1}{i \mkern1mu \omega \hspace{0.15ex} C}
	\label{eqn:impedanz}
\end{equation}
zusammengefasst werden. Dabei entspricht der reale Term $Z_{\hspace{-0.15ex}R}$ dem verbauten ohmschen
Wirkwiderstand. Der Blindwiderstand des Kondensators ist mit $Z_C$ \mbox{angegeben. Weiter gilt}
\begin{equation}
	U \hspace{-0.025ex} = ZI
	\label{eqn:ohm_komplex}
\end{equation}
als verallgemeinerter Zusammenhang zwischen Spannung, Impedanz und Stromstärke. Speziell folgen
aus \eqref{eqn:ohm_komplex} die Proportionalitäten
$U_{\hspace{-0.025ex}\text{in}\!} \propto Z_{\hspace{-0.15ex}RC}\hspace{0.15ex}$ für die
Eingangsspannung und $U_{\hspace{-0.025ex}\text{out}\!} \propto Z_C\hspace{0.15ex}$ für jene Spannung,
die am Kondensator abfällt. Durch \eqref{eqn:impedanz} lässt sich mit
\begin{equation}
	H = \pfrac{U_{\hspace{-0.025ex}\text{out}}}{U_{\hspace{-0.025ex}\text{in}\!}} = 
	\pfrac{Z_C}{Z_{\hspace{-0.15ex}RC}} = \pfrac{1}{1 + i \mkern1mu \omega R \hspace{0.15ex} C}
	\label{eqn:transmission}
\end{equation}
schließlich das Übertragungsverhältnis aufstellen. 

\subsubsection{Amplitudenverlauf}

Aus Gleichung \eqref{eqn:transmission} ist die betragsmäßige Transmission direkt mit dem Ausdruck
\begin{equation}
	\left|H\hspace{0.2ex}\right| =
	\left|\pfrac{U_{\hspace{-0.025ex}\text{out}}}{U_{\hspace{-0.025ex}\text{in}\!}}\right| = 
	\pfrac{\left|Z_C\right|}{\left|Z_{\hspace{-0.15ex}RC}\right|} =
	\pfrac{1}{\sqrt{1 + \omega^2 \hspace{-0.2ex} R^2 C^2}}
\end{equation}
bestimmt. Die Amplitude $U\hspace{-0.025ex}$ der Kondensatorspannung verläuft dann nach der \mbox{Beziehung}
\begin{equation}
	U\hspace{-0.025ex}(\omega) = \pfrac{U_{\hspace{-0.025ex}0}}{\sqrt{1 + \omega^2 \hspace{-0.2ex} R^2 C^2}}
	\label{eqn:amplitude_frequenz}
\end{equation}
in Abhängigkeit zur Kreisfrequenz $\omega$. Demnach bildet der RC\hspace{0.15ex}-Kreis einen Tiefpass.

\subsubsection{Phasenverschiebung}

Konjugiertes Ergänzen von \eqref{eqn:transmission} zeigt, dass zusätzlich die Proportionalitätsrelation
\begin{equation}
	H \hspace{-0.15ex} \propto 1 - \hspace{0.15ex} i \mkern1mu \omega R \hspace{0.15ex} C
	\label{eqn:proportion}
\end{equation}
erfüllt ist. Indem aus der Gaußschen Zahlenebene der geometrische Zusammenhang
\begin{equation}
	\tan(\varphi) = \pfrac{\sin(\varphi)}{\cos(\varphi)}
	= \pfrac{\operatorname{Im}(\hspace{-0.1ex}H\hspace{0.1ex})}{\operatorname{Re}(\hspace{-0.1ex}H\hspace{0.1ex})}
	= - \hspace{0.15ex} \omega R \hspace{0.15ex} C
	\label{eqn:phase_tangens}
\end{equation}
entnommen wird, lässt sich so die Phase $\varphi = \operatorname{arg}(\hspace{-0.1ex}H\hspace{0.1ex})$
zwischen $U_{\hspace{-0.025ex}\text{in}\!}$ und $U_{\hspace{-0.025ex}\text{out}\!}$ als
\begin{equation}
	\varphi(\omega) = - \arctan ( \omega R \hspace{0.15ex} C \hspace{0.1ex})
	\label{eqn:phase}
\end{equation}
definieren. Schließlich folgt nach Einsetzen von \eqref{eqn:phase_tangens} in \eqref{eqn:amplitude_frequenz} noch
\begin{equation}
	U\hspace{-0.025ex}(\varphi) = \pfrac{\cos(\varphi)}{\sqrt{\cos^2(\varphi) + \sin^2(\varphi)}}
	\, U_{\hspace{-0.025ex}0} = \cos(\varphi) \, U_{\hspace{-0.025ex}0}
	\label{eqn:amplitude_phase}
\end{equation}
als phasenabhängiger Ausdruck für die Amplitudenspannung. 

\subsection{Integrationsanwendung}

Bei Eingangsspannungen mit $\omega \gg \tau^{-1}$ läuft $Z_C$ gegen Null. Dann kann die Näherung
\begin{equation}
	U_{\hspace{-0.025ex}\text{in}\!} = ZI = Z_{\hspace{-0.15ex}R} I = RI = R \, \pfrac{\symup dQ}{\symup dt} =
	RC \: \pfrac{\symup dU_{\hspace{-0.025ex}\text{out}\!}}{\symup dt}
	\stepcounter{equation}\tag{18a}
\end{equation}
aus \eqref{eqn:ohm_komplex} hergeleitet werden. Integrieren von $0$ bis $t$ liefert die Beziehung
\begin{equation}
	U_{\hspace{-0.025ex}\text{out}\!}= \pfrac{1}{\hspace{-0.15ex}RC\hspace{0.15ex}} \!
	\int\limits_0^t \! U_{\hspace{-0.025ex}\text{in}\!} \: \symup dt'
	\tag{18b}
	\label{eqn:integral}
\end{equation}
für den RC\hspace{0.15ex}-Kreis. Dieser beschreibt somit einen Integrator hochfrequenter Signale.

\newpage

\subsection{Fehlerrechnung}

Um die Abweichung der Messgrößen zu untersuchen, werden noch einige weitere Formeln benötigt.
Die Rechenvorschrift für den Mittelwert $\overline{x}$ ist mit
\begin{equation}
	\overline{x} = \pfrac{1}{N \,} \sum_{n=1}^N x_n
	\label{eqn:mittel}
\end{equation}
gegeben. Zur Bestimmung der Standardabweichung $\symup{\Delta}\overline{x}$ kann
\begin{equation}
	(\symup{\Delta}\overline{x})^2 = \pfrac{1}{N(N-1)} \sum_{n=1}^N (x_n \! - \overline{x})^2
	\label{eqn:std}
\end{equation}
verwendet werden. Durch die Gaußsche Fehlerfortpflanzung
\begin{equation}
	(\symup{\Delta}f)^2 = \sum_{n=1}^N
	\left( \! \pfrac{\partial^{\!} f}{\partial x_{\raisebox{0.2ex}{$\scriptstyle{n}$}}} \!
	\right)^{\!\! 2} \!\! (\symup{\Delta}x_{\raisebox{0.2ex}{$\scriptstyle{n}$}})^2
	\label{eqn:gauss}
\end{equation}
ist die Abweichung $\symup{\Delta}f$ für von fehlerbehafteten Werten $x_n \!$ abhängige
Größen $\hspace{-0.2ex} f \hspace{0.2ex}$ definiert.

\subsection{Lineare Regression}

Entlang der linear abhängigen Messgrößen $(x_n,y_n)$ für $n = 1,\dotsc,\hspace{-0.1ex}N\hspace{0.1ex}$
lassen sich über
\begin{align}
	m = \pfrac{\overline{x\hspace{0.15ex}y\hspace{0.15ex}} - \overline{x} \: \overline{y\hspace{0.15ex}}}
	{\overline{x^2} - \overline{x}^{\hspace{0.15ex}2}} && b = \overline{y\hspace{0.15ex}} - m \overline{x}
	\label{eqn:regression}
\end{align}
die Parameter der Ausgleichsgerade $y = mx + b$ berechnen.
