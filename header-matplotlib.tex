% essential math commands
\usepackage{amsmath}
% many math symbols
\usepackage{amssymb}
% extensions for amsmath
\usepackage{mathtools}
% font settings
\usepackage{fontspec} % loads Latin Modern Fonts automatically

\usepackage[
	math-style=ISO,		% ┐
	bold-style=ISO,		% │
	sans-style=italic,	% │ follow iso standard
	nabla=upright,		% │
	partial=upright,	% ┘
	warnings-off={				% ┐
		mathtools-colon,		% │ remove redundant warnings
		mathtools-overbracket,	% ┘
	},
]{unicode-math}

% traditional math fonts
\setmathfont{Latin Modern Math}
% \setmathfont{XITS Math}[range={scr, bfscr}]
% \setmathfont{XITS Math}[range={cal, bfcal}, StylisticSet=1]

% numbers and units num, unit, qty, ang
\usepackage[
	locale=DE,					% german settings
	separate-uncertainty=true,	% always use pm for uncertainties
	per-mode=reciprocal,		% negative power for inverse units
	output-decimal-marker=.,	% use . instead of , for decimals
]{siunitx}

% chemical formulas
\usepackage[
	version=4,
	math-greek=default,	% ┐ allow functioning with unicode-math
	text-greek=default,	% ┘
]{mhchem}

% additional fraction type sfrac to frac, dfrac, tfrac, cfrac
\usepackage{xfrac}

% input useful macros
% Hammerite, https://tex.stackexchange.com/a/257122

\usepackage{pgfmath, xparse}

\newlength{\MathStrutDepth}
\newlength{\MathStrutHeight}
\settoheight{\MathStrutHeight}{$\mathstrut$}
\settodepth{\MathStrutDepth}{$\mathstrut$}

\newlength{\NumeratorDepth}
\newlength{\DenominatorHeight}
\newlength{\DepthNegativeDifference}
\newlength{\HeightPositiveDifference}
\newlength{\NumeratorBaselineCorrection}
\newlength{\DenominatorBaselineCorrection}

\newlength{\AdditionalEVSFracVerticalSpacing}
\setlength{\AdditionalEVSFracVerticalSpacing}{0.05mm}

% Fraction with equal top-and-bottom vertical spacing around the bar.
% Suited only to simple fractions that do not appear near other fractions.
% When used alongside other fractions, numerator and denominator baselines
% might not be aligned, which might give ugly results.
% Additionally, the default line thickness for overlines and fractions is restored.
\NewDocumentCommand\pfrac{omom}{%
% 		\Umathfractionrule\displaystyle=0.4pt\relax
% 		\Umathoverbarrule\displaystyle=0.4pt\relax
% 		\Umathoverbarvgap\displaystyle=1.4pt\relax
% 		\Umathfractionrule\textstyle=0.4pt\relax
% 		\Umathoverbarrule\textstyle=0.4pt\relax
% 		\Umathoverbarvgap\textstyle=1.4pt\relax
% 		\Umathfractionrule\crampeddisplaystyle=0.4pt\relax
% 		\Umathoverbarrule\crampeddisplaystyle=0.4pt\relax
% 		\Umathoverbarvgap\crampeddisplaystyle=1.4pt\relax
% 		\Umathfractionrule\crampedtextstyle=0.4pt\relax
% 		\Umathoverbarrule\crampedtextstyle=0.4pt\relax
% 		\Umathoverbarvgap\crampedtextstyle=1.4pt\relax
    \IfValueTF{#1}%
              {\settodepth{\NumeratorDepth}{$#1$}}%
              {\settodepth{\NumeratorDepth}{$#2$}}%
    \IfValueTF{#3}%
              {\settoheight{\DenominatorHeight}{$#3$}}%
              {\settoheight{\DenominatorHeight}{$#4$}}%
    \pgfmathsetlength%
        {\DepthNegativeDifference}%
        {\NumeratorDepth - \MathStrutDepth}%
    \pgfmathsetlength%
        {\HeightPositiveDifference}%
        {\MathStrutHeight - \DenominatorHeight}%
    \pgfmathsetlength%
        {\NumeratorBaselineCorrection}%
        {\AdditionalEVSFracVerticalSpacing + \DepthNegativeDifference + \HeightPositiveDifference}%
    \pgfmathsetlength%
        {\DenominatorBaselineCorrection}%
        {-\AdditionalEVSFracVerticalSpacing}%
    \def\Numerator{\raisebox{\NumeratorBaselineCorrection}{$#2$}}%
    \def\Denominator{\raisebox{\DenominatorBaselineCorrection}{$#4$}}%
    \frac{\Numerator}{\Denominator}%
}
 % fraction type with equal v and h spacing pfrac
